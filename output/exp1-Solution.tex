\documentclass[final]{article}
\usepackage{amssymb,amsmath,amsfonts}
\usepackage[dvips]{graphicx}
\usepackage{longtable}
\usepackage{verbatim}
\usepackage{fancyhdr}
\usepackage{pstricks-add,pst-slpe}
\pagestyle{fancy}
\fancyhead{}
\fancyfoot{}
\fancyhead[CO,CE]{$z_{1}^{-1}c_{1}z_{1}c_{1}^{-1}=_F 1$}
\fancyfoot[RO, LE] {\thepage}
\begin{document}
\date{}
\title{
  {\Large Solution of the equation \\$z_{1}^{-1}c_{1}z_{1}c_{1}^{-1}=_F 1$\\ in a Free Group}
  {\normalsize
   \author{Bilal Khan
        \thanks{Department of Mathematics and Computer Science, John Jay College of Criminal Justice, City University of New York (CUNY).}
   \and M-K Solver
        \thanks{This report was generated automatically by software developed with support from the National Security Agency Grant H98230-06-1-0042.}
           }
  }
}

\maketitle

\tableofcontents

\newpage
\section{Cancellation scheme \#$1$}
\begin{center}
\begin{pspicture}(-0.5,-0.5)(6.5,6.5)
{\psset{fillstyle=ccslope,slopebegin=yellow!40,slopeend=gray}
\cnodeput(2.98,3.00){0}{\strut\boldmath$0$}
\cnodeput(6.00,0.00){1}{\strut\boldmath$1$}
\cnodeput(0.00,6.00){2}{\strut\boldmath$2$}
}
\newcommand\arc[3]{%
  \ncline{#1}{#2}{#3}
}
\arc{-}{0}{2}{}
\arc{-}{0}{1}{}
\psline[linecolor=red]{<<-|}(5.82,-0.18)(4.32,1.32)(2.81,2.82)\rput{315}(4.32,1.32){$z_{1}$}
\psline[linecolor=blue]{|->>}(3.16,3.18)(4.67,1.68)(6.18,0.18)\rput{315}(4.67,1.68){$c_{1}$}
\psline[linecolor=red]{|->>}(0.18,6.18)(1.67,4.68)(3.16,3.17)\rput{314}(1.67,4.68){$z_{1}$}
\psline[linecolor=blue]{<<-|}(2.81,2.82)(1.31,4.32)(-0.18,5.82)\rput{314}(1.31,4.32){$c_{1}$}
\end{pspicture}
\end{center}
\begin{center}
\begin{tabular}{|ll|}
\hline
$z_{1}^{-1}$ & $1\leftarrow 0$\\
$c_{1}$ & $0\leftarrow 1$\\
$z_{1}$ & $2\leftarrow 0$\\
$c_{1}^{-1}$ & $0\leftarrow 2$\\
\hline
\end{tabular}
\end{center}
\subsection*{Generalized Equation root-1}
\label{root-1}Below is the root GE obtained from the cancellation diagram above.\begin{center}
\begin{pspicture}(-0.5,-0.5)(7.5,6.5)
\psline[linecolor=black]{-}(0.0,0.0)(0.0,6.0)\rput{0}(0.0,0.0){$0$}
\rput{0}(0.0,6.0){$0$}
\psline[linecolor=black]{-}(1.75,0.0)(1.75,6.0)\rput{0}(1.75,0.0){$1$}
\rput{0}(1.75,6.0){$1$}
\psline[linecolor=black]{-}(3.5,0.0)(3.5,6.0)\rput{0}(3.5,0.0){$2$}
\rput{0}(3.5,6.0){$2$}
\psline[linecolor=black]{-}(5.25,0.0)(5.25,6.0)\rput{0}(5.25,0.0){$3$}
\rput{0}(5.25,6.0){$3$}
\psline[linecolor=black]{-}(7.0,0.0)(7.0,6.0)\rput{0}(7.0,0.0){$4$}
\rput{0}(7.0,6.0){$4$}
\psline[linecolor=blue]{[->}(1.75,1.0)(3.5,1.0)\rput{0}(2.625,0.8){$c_{1}$}
\psline[linecolor=red]{[->}(3.5,2.0)(5.25,2.0)\rput{0}(4.375,1.8){$z_{1}$}
\psline[linecolor=red]{<-]}(0.0,2.0)(1.75,2.0)\rput{0}(0.875,1.8){$z_{1}$}
\pscircle[linecolor=red,fillcolor=black,fillstyle=solid](0.0,2.0){0.075}
\pscircle[linecolor=red,fillcolor=black,fillstyle=solid](5.25,2.0){0.075}
\pscircle[linecolor=red,fillcolor=white,fillstyle=solid](1.75,2.0){0.075}
\pscircle[linecolor=red,fillcolor=white,fillstyle=solid](3.5,2.0){0.075}
\psline[linecolor=blue]{<-]}(5.25,1.0)(7.0,1.0)\rput{0}(6.125,0.8){$c_{1}$}
\psline[linecolor=red]{[->}(3.5,3.0)(5.25,3.0)\rput{0}(4.375,2.8){$z_{100}$}
\pscircle[linecolor=red,fillcolor=black,fillstyle=solid](3.5,3.0){0.075}
\pscircle[linecolor=red,fillcolor=black,fillstyle=solid](7.0,4.0){0.075}
\pscircle[linecolor=red,fillcolor=white,fillstyle=solid](5.25,3.0){0.075}
\pscircle[linecolor=red,fillcolor=white,fillstyle=solid](5.25,4.0){0.075}
\psline[linecolor=red]{<-]}(5.25,4.0)(7.0,4.0)\rput{0}(6.125,3.8){$z_{100}$}
\psline[linecolor=red]{[->}(0.0,3.0)(1.75,3.0)\rput{0}(0.875,2.8){$z_{101}$}
\pscircle[linecolor=red,fillcolor=black,fillstyle=solid](0.0,3.0){0.075}
\pscircle[linecolor=red,fillcolor=black,fillstyle=solid](3.5,4.0){0.075}
\pscircle[linecolor=red,fillcolor=white,fillstyle=solid](1.75,3.0){0.075}
\pscircle[linecolor=red,fillcolor=white,fillstyle=solid](1.75,4.0){0.075}
\psline[linecolor=red]{<-]}(1.75,4.0)(3.5,4.0)\rput{0}(2.625,3.8){$z_{101}$}
\end{pspicture}
\end{center}
{\bf GE Information}:  
Carrier: [0-1:z1-.] ;  
Carrier Dual: [2-3:z1+.] ;  
Critical Boundary: 1;  
The GE above is non-degenerate.  This GE is {\em not} a leaf in the GE tree.   It has 1 valid prints (descendents).  \\[0.1in]
   It has 1 legal carrier-to-dual prints, as follows:
\begin{verbatim}
     Print 1: =0=3*<1=2*
\end{verbatim}
We proceed.\\[0.2in]
\subsection*{Generalized Equation root-1.1}
\label{root-1.1}We begin from the GE root-1 (see pp. \pageref{root-1}).  {We consider its print}
\begin{verbatim}
     Print 1: =0=3*<1=2*
\end{verbatim}
{\bf Sequence of actions in performing the Print 1:}\\
{\underline{Step 1}:} Moved (old) base [0-1:z1-.]  to (new) boundaries 3 - 2.\\
{\underline{Step 2}:} Moved (old) base [0-1:z101+.]  to (new) boundaries 3 - 2.\\
{\underline{Step 3}:} Collapsed (new) base [2-3:z1+.]  to the empty base (3,3).
\\
{\underline{Step 4}:} Deleted (new) boundary 0 because it is not used inside any base.  This will cause renumbering of higher numbered boundaries.
\\[0.1in]
{Upon applying the print, the GE we obtain---which we refer to as root-1.1---is illustrated below:}
\begin{center}
\begin{pspicture}(-0.5,-0.5)(7.5,6.5)
\psline[linecolor=black]{-}(0.0,0.0)(0.0,6.0)\rput{0}(0.0,0.0){$0$}
\rput{0}(0.0,6.0){$0$}
\psline[linecolor=black]{-}(2.3333333333333335,0.0)(2.3333333333333335,6.0)\rput{0}(2.3333333333333335,0.0){$1$}
\rput{0}(2.3333333333333335,6.0){$1$}
\psline[linecolor=black]{-}(4.666666666666667,0.0)(4.666666666666667,6.0)\rput{0}(4.666666666666667,0.0){$2$}
\rput{0}(4.666666666666667,6.0){$2$}
\psline[linecolor=black]{-}(7.0,0.0)(7.0,6.0)\rput{0}(7.0,0.0){$3$}
\rput{0}(7.0,6.0){$3$}
\psline[linecolor=blue]{[->}(0.0,0.75)(2.3333333333333335,0.75)\rput{0}(1.1666666666666667,0.55){$c_{1}$}
\psline[linecolor=red]{[->}(4.666666666666667,0.75)(4.666666666666667,0.75)\rput{0}(4.666666666666667,0.55){$z_{1}$}
\pscircle[linecolor=red,fillcolor=black,fillstyle=solid](4.666666666666667,0.75){0.075}
\pscircle[linecolor=red,fillcolor=black,fillstyle=solid](4.666666666666667,1.5){0.075}
\psline[linecolor=red]{[->}(4.666666666666667,1.5)(4.666666666666667,1.5)\rput{0}(4.666666666666667,1.3){$z_{1}$}
\psline[linecolor=blue]{<-]}(4.666666666666667,2.25)(7.0,2.25)\rput{0}(5.833333333333334,2.05){$c_{1}$}
\psline[linecolor=red]{[->}(2.3333333333333335,3.0)(4.666666666666667,3.0)\rput{0}(3.5,2.8){$z_{100}$}
\pscircle[linecolor=red,fillcolor=black,fillstyle=solid](2.3333333333333335,3.0){0.075}
\pscircle[linecolor=red,fillcolor=black,fillstyle=solid](7.0,3.75){0.075}
\pscircle[linecolor=red,fillcolor=white,fillstyle=solid](4.666666666666667,3.0){0.075}
\pscircle[linecolor=red,fillcolor=white,fillstyle=solid](4.666666666666667,3.75){0.075}
\psline[linecolor=red]{<-]}(4.666666666666667,3.75)(7.0,3.75)\rput{0}(5.833333333333334,3.55){$z_{100}$}
\psline[linecolor=red]{<-]}(2.3333333333333335,4.5)(4.666666666666667,4.5)\rput{0}(3.5,4.3){$z_{101}$}
\psline[linecolor=red]{<-]}(0.0,1.5)(2.3333333333333335,1.5)\rput{0}(1.1666666666666667,1.3){$z_{101}$}
\pscircle[linecolor=red,fillcolor=black,fillstyle=solid](0.0,1.5){0.075}
\pscircle[linecolor=red,fillcolor=black,fillstyle=solid](2.3333333333333335,4.5){0.075}
\pscircle[linecolor=red,fillcolor=white,fillstyle=solid](2.3333333333333335,1.5){0.075}
\pscircle[linecolor=red,fillcolor=white,fillstyle=solid](4.666666666666667,4.5){0.075}
\end{pspicture}
\end{center}
{\bf GE Information}:  
Carrier: [0-1:z101-.] ;  
Carrier Dual: [1-2:z101-.] ;  
Critical Boundary: 1;  
The GE above is non-degenerate.  This GE is {\em not} a leaf in the GE tree.   It has 1 valid prints (descendents).  \\[0.1in]
   It has 1 legal carrier-to-dual prints, as follows:
\begin{verbatim}
     Print 1: =0=1*<1=2*
\end{verbatim}
This completes the consideration of root-1.1, as derived from the application of a print to root-1.\\[0.1in]
\subsection*{Generalized Equation root-1.1.1}
\label{root-1.1.1}We begin from the GE root-1.1 (see pp. \pageref{root-1.1}).  {We consider its print}
\begin{verbatim}
     Print 1: =0=1*<1=2*
\end{verbatim}
{\bf Sequence of actions in performing the Print 1:}\\
{\underline{Step 1}:} Moved (old) base [0-1:z101-.]  to (new) boundaries 1 - 2.\\
{\underline{Step 2}:} Moved (old) base [0-1:c1+.]  to (new) boundaries 1 - 2.\\
{\underline{Step 3}:} Collapsed (new) base [1-2:z101-.]  to the empty base (2,2).
\\
{\underline{Step 4}:} Deleted (new) boundary 0 because it is not used inside any base.  This will cause renumbering of higher numbered boundaries.
\\[0.1in]
{Upon applying the print, the GE we obtain---which we refer to as root-1.1.1---is illustrated below:}
\begin{center}
\begin{pspicture}(-0.5,-0.5)(7.5,6.5)
\psline[linecolor=black]{-}(0.0,0.0)(0.0,6.0)\rput{0}(0.0,0.0){$0$}
\rput{0}(0.0,6.0){$0$}
\psline[linecolor=black]{-}(3.5,0.0)(3.5,6.0)\rput{0}(3.5,0.0){$1$}
\rput{0}(3.5,6.0){$1$}
\psline[linecolor=black]{-}(7.0,0.0)(7.0,6.0)\rput{0}(7.0,0.0){$2$}
\rput{0}(7.0,6.0){$2$}
\psline[linecolor=blue]{[->}(0.0,0.6)(3.5,0.6)\rput{0}(1.75,0.39999999999999997){$c_{1}$}
\psline[linecolor=red]{[->}(3.5,1.2)(3.5,1.2)\rput{0}(3.5,1.0){$z_{1}$}
\psline[linecolor=red]{[->}(3.5,1.7999999999999998)(3.5,1.7999999999999998)\rput{0}(3.5,1.5999999999999999){$z_{1}$}
\pscircle[linecolor=red,fillcolor=black,fillstyle=solid](3.5,1.7999999999999998){0.075}
\pscircle[linecolor=red,fillcolor=black,fillstyle=solid](3.5,1.2){0.075}
\psline[linecolor=blue]{<-]}(3.5,2.4)(7.0,2.4)\rput{0}(5.25,2.1999999999999997){$c_{1}$}
\psline[linecolor=red]{[->}(0.0,3.0)(3.5,3.0)\rput{0}(1.75,2.8){$z_{100}$}
\pscircle[linecolor=red,fillcolor=black,fillstyle=solid](0.0,3.0){0.075}
\pscircle[linecolor=red,fillcolor=black,fillstyle=solid](7.0,3.5999999999999996){0.075}
\pscircle[linecolor=red,fillcolor=white,fillstyle=solid](3.5,3.0){0.075}
\pscircle[linecolor=red,fillcolor=white,fillstyle=solid](3.5,3.5999999999999996){0.075}
\psline[linecolor=red]{<-]}(3.5,3.5999999999999996)(7.0,3.5999999999999996)\rput{0}(5.25,3.3999999999999995){$z_{100}$}
\psline[linecolor=red]{<-]}(3.5,4.2)(3.5,4.2)\rput{0}(3.5,4.0){$z_{101}$}
\psline[linecolor=red]{<-]}(3.5,4.8)(3.5,4.8)\rput{0}(3.5,4.6){$z_{101}$}
\pscircle[linecolor=red,fillcolor=black,fillstyle=solid](3.5,4.8){0.075}
\pscircle[linecolor=red,fillcolor=black,fillstyle=solid](3.5,4.2){0.075}
\end{pspicture}
\end{center}
{\bf GE Information}:  
Carrier: [0-1:z100+.] ;  
Carrier Dual: [1-2:z100-.] ;  
Critical Boundary: 1;  
The GE above is non-degenerate.  This GE is {\em not} a leaf in the GE tree.   It has 1 valid prints (descendents).  \\[0.1in]
   It has 1 legal carrier-to-dual prints, as follows:
\begin{verbatim}
     Print 1: =0=2*<1=1*
\end{verbatim}
This completes the consideration of root-1.1.1, as derived from the application of a print to root-1.1.\\[0.1in]
\subsection*{Generalized Equation root-1.1.1.1}
\label{root-1.1.1.1}We begin from the GE root-1.1.1 (see pp. \pageref{root-1.1.1}).  {We consider its print}
\begin{verbatim}
     Print 1: =0=2*<1=1*
\end{verbatim}
{\bf Sequence of actions in performing the Print 1:}\\
{\underline{Step 1}:} Moved (old) base [0-1:z100+.]  to (new) boundaries 2 - 1.\\
{\underline{Step 2}:} Moved (old) base [0-1:c1+.]  to (new) boundaries 2 - 1.\\
{\underline{Step 3}:} Collapsed (new) base [1-2:z100-.]  to the empty base (2,2).
\\
{\underline{Step 4}:} Deleted (new) boundary 0 because it is not used inside any base.  This will cause renumbering of higher numbered boundaries.
\\[0.1in]
{Upon applying the print, the GE we obtain---which we refer to as root-1.1.1.1---is illustrated below:}
\begin{center}
\begin{pspicture}(-0.5,-0.5)(7.5,6.5)
\psline[linecolor=black]{-}(0.0,0.0)(0.0,6.0)\rput{0}(0.0,0.0){$0$}
\rput{0}(0.0,6.0){$0$}
\psline[linecolor=black]{-}(7.0,0.0)(7.0,6.0)\rput{0}(7.0,0.0){$1$}
\rput{0}(7.0,6.0){$1$}
\psline[linecolor=blue]{<-]}(0.0,0.75)(7.0,0.75)\rput{0}(3.5,0.55){$c_{1}$}
\psline[linecolor=red]{[->}(0.0,1.5)(0.0,1.5)\rput{0}(0.0,1.3){$z_{1}$}
\psline[linecolor=red]{[->}(0.0,2.25)(0.0,2.25)\rput{0}(0.0,2.05){$z_{1}$}
\pscircle[linecolor=red,fillcolor=black,fillstyle=solid](0.0,2.25){0.075}
\pscircle[linecolor=red,fillcolor=black,fillstyle=solid](0.0,1.5){0.075}
\psline[linecolor=blue]{<-]}(0.0,3.0)(7.0,3.0)\rput{0}(3.5,2.8){$c_{1}$}
\psline[linecolor=red]{<-]}(7.0,1.5)(7.0,1.5)\rput{0}(7.0,1.3){$z_{100}$}
\psline[linecolor=red]{<-]}(7.0,2.25)(7.0,2.25)\rput{0}(7.0,2.05){$z_{100}$}
\pscircle[linecolor=red,fillcolor=black,fillstyle=solid](7.0,2.25){0.075}
\pscircle[linecolor=red,fillcolor=black,fillstyle=solid](7.0,1.5){0.075}
\psline[linecolor=red]{<-]}(0.0,3.75)(0.0,3.75)\rput{0}(0.0,3.55){$z_{101}$}
\psline[linecolor=red]{<-]}(0.0,4.5)(0.0,4.5)\rput{0}(0.0,4.3){$z_{101}$}
\pscircle[linecolor=red,fillcolor=black,fillstyle=solid](0.0,4.5){0.075}
\pscircle[linecolor=red,fillcolor=black,fillstyle=solid](0.0,3.75){0.075}
\end{pspicture}
\end{center}
{\bf GE Information}:  
Carrier: [0-0:z1+.] ;  
Carrier Dual: [0-0:z1+.] ;  
Critical Boundary: 0;  
The GE above is non-degenerate.  This GE is a leaf in the GE tree.  We have effectively found a solution!\\[0.1in]
This completes the consideration of root-1.1.1.1, as derived from the application of a print to root-1.1.1.\\[0.1in]
\newpage
\section{Cancellation scheme \#$2$}
\begin{center}
\begin{pspicture}(-0.5,-0.5)(6.5,6.5)
{\psset{fillstyle=ccslope,slopebegin=yellow!40,slopeend=gray}
\cnodeput(1.92,1.98){0}{\strut\boldmath$0$}
\cnodeput(3.92,4.02){1}{\strut\boldmath$1$}
\cnodeput(6.00,6.00){2}{\strut\boldmath$2$}
\cnodeput(0.00,0.00){3}{\strut\boldmath$3$}
}
\newcommand\arc[3]{%
  \ncline{#1}{#2}{#3}
}
\arc{-}{0}{3}{}
\arc{-}{0}{1}{}
\arc{-}{1}{2}{}
\psline[linecolor=red]{<<-|}(4.09,3.85)(3.10,2.82)(2.10,1.80)\rput{45}(3.10,2.82){$z_{1}$}
\psline[linecolor=blue]{|->>}(6.17,5.82)(5.13,4.83)(4.09,3.84)\rput{43}(5.13,4.83){$c_{1}$}
\pscurve[linecolor=red]{|->>}(-0.18,0.17)(0.78,1.16)(2.74,3.17)(3.74,4.19)(4.79,5.19)(5.83,6.18)\rput{45}(2.74,3.17){$z_{1}$}
\psline[linecolor=blue]{<<-|}(2.10,1.80)(1.14,0.81)(0.18,-0.17)\rput{45}(1.14,0.81){$c_{1}$}
\end{pspicture}
\end{center}
\begin{center}
\begin{tabular}{|ll|}
\hline
$z_{1}^{-1}$ & $1\leftarrow 0$\\
$c_{1}$ & $2\leftarrow 1$\\
$z_{1}$ & $3\leftarrow 0\leftarrow 1\leftarrow 2$\\
$c_{1}^{-1}$ & $0\leftarrow 3$\\
\hline
\end{tabular}
\end{center}
\subsection*{Generalized Equation root-2}
\label{root-2}Below is the root GE obtained from the cancellation diagram above.\begin{center}
\begin{pspicture}(-0.5,-0.5)(7.5,6.5)
\psline[linecolor=black]{-}(0.0,0.0)(0.0,6.0)\rput{0}(0.0,0.0){$0$}
\rput{0}(0.0,6.0){$0$}
\psline[linecolor=black]{-}(1.1666666666666667,0.0)(1.1666666666666667,6.0)\rput{0}(1.1666666666666667,0.0){$1$}
\rput{0}(1.1666666666666667,6.0){$1$}
\psline[linecolor=black]{-}(2.3333333333333335,0.0)(2.3333333333333335,6.0)\rput{0}(2.3333333333333335,0.0){$2$}
\rput{0}(2.3333333333333335,6.0){$2$}
\psline[linecolor=black]{-}(3.5,0.0)(3.5,6.0)\rput{0}(3.5,0.0){$3$}
\rput{0}(3.5,6.0){$3$}
\psline[linecolor=black]{-}(4.666666666666667,0.0)(4.666666666666667,6.0)\rput{0}(4.666666666666667,0.0){$4$}
\rput{0}(4.666666666666667,6.0){$4$}
\psline[linecolor=black]{-}(5.833333333333334,0.0)(5.833333333333334,6.0)\rput{0}(5.833333333333334,0.0){$5$}
\rput{0}(5.833333333333334,6.0){$5$}
\psline[linecolor=black]{-}(7.000000000000001,0.0)(7.000000000000001,6.0)\rput{0}(7.000000000000001,0.0){$6$}
\rput{0}(7.000000000000001,6.0){$6$}
\psline[linecolor=blue]{[->}(1.1666666666666667,1.0)(2.3333333333333335,1.0)\rput{0}(1.75,0.8){$c_{1}$}
\psline[linecolor=red]{[->}(2.3333333333333335,2.0)(5.833333333333334,2.0)\rput{0}(4.083333333333334,1.8){$z_{1}$}
\psline[linecolor=red]{<-]}(0.0,2.0)(1.1666666666666667,2.0)\rput{0}(0.5833333333333334,1.8){$z_{1}$}
\pscircle[linecolor=red,fillcolor=black,fillstyle=solid](0.0,2.0){0.075}
\pscircle[linecolor=red,fillcolor=black,fillstyle=solid](5.833333333333334,2.0){0.075}
\pscircle[linecolor=red,fillcolor=white,fillstyle=solid](1.1666666666666667,2.0){0.075}
\pscircle[linecolor=red,fillcolor=white,fillstyle=solid](2.3333333333333335,2.0){0.075}
\psline[linecolor=blue]{<-]}(5.833333333333334,1.0)(7.0,1.0)\rput{0}(6.416666666666667,0.8){$c_{1}$}
\psline[linecolor=red]{[->}(4.666666666666667,3.0)(5.833333333333334,3.0)\rput{0}(5.25,2.8){$z_{100}$}
\pscircle[linecolor=red,fillcolor=black,fillstyle=solid](4.666666666666667,3.0){0.075}
\pscircle[linecolor=red,fillcolor=black,fillstyle=solid](7.0,4.0){0.075}
\pscircle[linecolor=red,fillcolor=white,fillstyle=solid](5.833333333333334,3.0){0.075}
\pscircle[linecolor=red,fillcolor=white,fillstyle=solid](5.833333333333334,4.0){0.075}
\psline[linecolor=red]{<-]}(5.833333333333334,4.0)(7.0,4.0)\rput{0}(6.416666666666667,3.8){$z_{100}$}
\psline[linecolor=red]{[->}(0.0,3.0)(1.1666666666666667,3.0)\rput{0}(0.5833333333333334,2.8){$z_{101}$}
\pscircle[linecolor=red,fillcolor=black,fillstyle=solid](0.0,3.0){0.075}
\pscircle[linecolor=red,fillcolor=black,fillstyle=solid](4.666666666666667,1.0){0.075}
\pscircle[linecolor=red,fillcolor=white,fillstyle=solid](1.1666666666666667,3.0){0.075}
\pscircle[linecolor=red,fillcolor=white,fillstyle=solid](3.5,1.0){0.075}
\psline[linecolor=red]{<-]}(3.5,1.0)(4.666666666666667,1.0)\rput{0}(4.083333333333334,0.8){$z_{101}$}
\psline[linecolor=red]{[->}(1.1666666666666667,4.0)(2.3333333333333335,4.0)\rput{0}(1.75,3.8){$z_{102}$}
\pscircle[linecolor=red,fillcolor=black,fillstyle=solid](1.1666666666666667,4.0){0.075}
\pscircle[linecolor=red,fillcolor=black,fillstyle=solid](3.5,3.0){0.075}
\pscircle[linecolor=red,fillcolor=white,fillstyle=solid](2.3333333333333335,4.0){0.075}
\pscircle[linecolor=red,fillcolor=white,fillstyle=solid](2.3333333333333335,3.0){0.075}
\psline[linecolor=red]{<-]}(2.3333333333333335,3.0)(3.5,3.0)\rput{0}(2.916666666666667,2.8){$z_{102}$}
\end{pspicture}
\end{center}
{\bf GE Information}:  
Carrier: [0-1:z1-.] ;  
Carrier Dual: [2-5:z1+.] ;  
Critical Boundary: 1;  
The GE above is non-degenerate.  This GE is {\em not} a leaf in the GE tree.   It has 1 valid prints (descendents).  \\[0.1in]
   It has 1 legal carrier-to-dual prints, as follows:
\begin{verbatim}
     Print 1: =0=5*<4*<3*<1=2*
\end{verbatim}
We proceed.\\[0.2in]
\subsection*{Generalized Equation root-2.1}
\label{root-2.1}We begin from the GE root-2 (see pp. \pageref{root-2}).  {We consider its print}
\begin{verbatim}
     Print 1: =0=5*<4*<3*<1=2*
\end{verbatim}
{\bf Sequence of actions in performing the Print 1:}\\
{\underline{Step 1}:} Moved (old) base [0-1:z1-.]  to (new) boundaries 5 - 2.\\
{\underline{Step 2}:} Moved (old) base [0-1:z101+.]  to (new) boundaries 5 - 2.\\
{\underline{Step 3}:} Collapsed (new) base [2-5:z1+.]  to the empty base (5,5).
\\
{\underline{Step 4}:} Deleted (new) boundary 0 because it is not used inside any base.  This will cause renumbering of higher numbered boundaries.
\\[0.1in]
{Upon applying the print, the GE we obtain---which we refer to as root-2.1---is illustrated below:}
\begin{center}
\begin{pspicture}(-0.5,-0.5)(7.5,6.5)
\psline[linecolor=black]{-}(0.0,0.0)(0.0,6.0)\rput{0}(0.0,0.0){$0$}
\rput{0}(0.0,6.0){$0$}
\psline[linecolor=black]{-}(1.4,0.0)(1.4,6.0)\rput{0}(1.4,0.0){$1$}
\rput{0}(1.4,6.0){$1$}
\psline[linecolor=black]{-}(2.8,0.0)(2.8,6.0)\rput{0}(2.8,0.0){$2$}
\rput{0}(2.8,6.0){$2$}
\psline[linecolor=black]{-}(4.199999999999999,0.0)(4.199999999999999,6.0)\rput{0}(4.199999999999999,0.0){$3$}
\rput{0}(4.199999999999999,6.0){$3$}
\psline[linecolor=black]{-}(5.6,0.0)(5.6,6.0)\rput{0}(5.6,0.0){$4$}
\rput{0}(5.6,6.0){$4$}
\psline[linecolor=black]{-}(7.0,0.0)(7.0,6.0)\rput{0}(7.0,0.0){$5$}
\rput{0}(7.0,6.0){$5$}
\psline[linecolor=blue]{[->}(0.0,0.75)(1.4,0.75)\rput{0}(0.7,0.55){$c_{1}$}
\psline[linecolor=red]{[->}(5.6,0.75)(5.6,0.75)\rput{0}(5.6,0.55){$z_{1}$}
\psline[linecolor=red]{[->}(5.6,1.5)(5.6,1.5)\rput{0}(5.6,1.3){$z_{1}$}
\pscircle[linecolor=red,fillcolor=black,fillstyle=solid](5.6,1.5){0.075}
\pscircle[linecolor=red,fillcolor=black,fillstyle=solid](5.6,0.75){0.075}
\psline[linecolor=blue]{<-]}(5.6,2.25)(7.0,2.25)\rput{0}(6.3,2.05){$c_{1}$}
\psline[linecolor=red]{[->}(4.199999999999999,3.0)(5.6,3.0)\rput{0}(4.8999999999999995,2.8){$z_{100}$}
\pscircle[linecolor=red,fillcolor=black,fillstyle=solid](4.199999999999999,3.0){0.075}
\pscircle[linecolor=red,fillcolor=black,fillstyle=solid](7.0,3.75){0.075}
\pscircle[linecolor=red,fillcolor=white,fillstyle=solid](5.6,3.0){0.075}
\pscircle[linecolor=red,fillcolor=white,fillstyle=solid](5.6,3.75){0.075}
\psline[linecolor=red]{<-]}(5.6,3.75)(7.0,3.75)\rput{0}(6.3,3.55){$z_{100}$}
\psline[linecolor=red]{<-]}(1.4,4.5)(5.6,4.5)\rput{0}(3.5,4.3){$z_{101}$}
\pscircle[linecolor=red,fillcolor=black,fillstyle=solid](1.4,4.5){0.075}
\pscircle[linecolor=red,fillcolor=black,fillstyle=solid](2.8,0.75){0.075}
\pscircle[linecolor=red,fillcolor=white,fillstyle=solid](5.6,4.5){0.075}
\pscircle[linecolor=red,fillcolor=white,fillstyle=solid](4.199999999999999,0.75){0.075}
\psline[linecolor=red]{<-]}(2.8,0.75)(4.199999999999999,0.75)\rput{0}(3.4999999999999996,0.55){$z_{101}$}
\psline[linecolor=red]{[->}(0.0,1.5)(1.4,1.5)\rput{0}(0.7,1.3){$z_{102}$}
\pscircle[linecolor=red,fillcolor=black,fillstyle=solid](0.0,1.5){0.075}
\pscircle[linecolor=red,fillcolor=black,fillstyle=solid](2.8,2.25){0.075}
\pscircle[linecolor=red,fillcolor=white,fillstyle=solid](1.4,1.5){0.075}
\pscircle[linecolor=red,fillcolor=white,fillstyle=solid](1.4,2.25){0.075}
\psline[linecolor=red]{<-]}(1.4,2.25)(2.8,2.25)\rput{0}(2.0999999999999996,2.05){$z_{102}$}
\end{pspicture}
\end{center}
{\bf GE Information}:  
Carrier: [0-1:z102+.] ;  
Carrier Dual: [1-2:z102-.] ;  
Critical Boundary: 1;  
Observe the following facts about this GE:
The base [1-4:z101-.]  and its dual are of the same polarity, yet one properly contains the other.  The base [2-3:z101-.]  and its dual are of the same polarity, yet one properly contains the other.  These observations show that the GE above is degenerate.  This GE is a leaf in the GE tree.  This branch of the tree has led us to a dead end.\\[0.1in]
This completes the consideration of root-2.1, as derived from the application of a print to root-2.\\[0.1in]
\newpage
\section{Cancellation scheme \#$3$}
\begin{center}
\begin{pspicture}(-0.5,-0.5)(6.5,6.5)
{\psset{fillstyle=ccslope,slopebegin=yellow!40,slopeend=gray}
\cnodeput(0.00,0.00){0}{\strut\boldmath$0$}
\cnodeput(3.05,2.99){1}{\strut\boldmath$1$}
\cnodeput(6.00,6.00){2}{\strut\boldmath$2$}
}
\newcommand\arc[3]{%
  \ncline{#1}{#2}{#3}
}
\arc{-}{0}{1}{}
\arc{-}{1}{2}{}
\psline[linecolor=red]{<<-|}(3.23,2.82)(1.70,1.32)(0.18,-0.18)\rput{44}(1.70,1.32){$z_{1}$}
\psline[linecolor=blue]{|->>}(6.18,5.82)(4.70,4.32)(3.23,2.82)\rput{45}(4.70,4.32){$c_{1}$}
\psline[linecolor=red]{|->>}(2.87,3.17)(4.35,4.67)(5.82,6.18)\rput{45}(4.35,4.67){$z_{1}$}
\psline[linecolor=blue]{<<-|}(-0.18,0.18)(1.35,1.68)(2.88,3.17)\rput{44}(1.35,1.68){$c_{1}$}
\end{pspicture}
\end{center}
\begin{center}
\begin{tabular}{|ll|}
\hline
$z_{1}^{-1}$ & $1\leftarrow 0$\\
$c_{1}$ & $2\leftarrow 1$\\
$z_{1}$ & $1\leftarrow 2$\\
$c_{1}^{-1}$ & $0\leftarrow 1$\\
\hline
\end{tabular}
\end{center}
\subsection*{Generalized Equation root-3}
\label{root-3}Below is the root GE obtained from the cancellation diagram above.\begin{center}
\begin{pspicture}(-0.5,-0.5)(7.5,6.5)
\psline[linecolor=black]{-}(0.0,0.0)(0.0,6.0)\rput{0}(0.0,0.0){$0$}
\rput{0}(0.0,6.0){$0$}
\psline[linecolor=black]{-}(1.75,0.0)(1.75,6.0)\rput{0}(1.75,0.0){$1$}
\rput{0}(1.75,6.0){$1$}
\psline[linecolor=black]{-}(3.5,0.0)(3.5,6.0)\rput{0}(3.5,0.0){$2$}
\rput{0}(3.5,6.0){$2$}
\psline[linecolor=black]{-}(5.25,0.0)(5.25,6.0)\rput{0}(5.25,0.0){$3$}
\rput{0}(5.25,6.0){$3$}
\psline[linecolor=black]{-}(7.0,0.0)(7.0,6.0)\rput{0}(7.0,0.0){$4$}
\rput{0}(7.0,6.0){$4$}
\psline[linecolor=blue]{[->}(1.75,1.0)(3.5,1.0)\rput{0}(2.625,0.8){$c_{1}$}
\psline[linecolor=red]{[->}(3.5,2.0)(5.25,2.0)\rput{0}(4.375,1.8){$z_{1}$}
\psline[linecolor=red]{<-]}(0.0,2.0)(1.75,2.0)\rput{0}(0.875,1.8){$z_{1}$}
\pscircle[linecolor=red,fillcolor=black,fillstyle=solid](0.0,2.0){0.075}
\pscircle[linecolor=red,fillcolor=black,fillstyle=solid](5.25,2.0){0.075}
\pscircle[linecolor=red,fillcolor=white,fillstyle=solid](1.75,2.0){0.075}
\pscircle[linecolor=red,fillcolor=white,fillstyle=solid](3.5,2.0){0.075}
\psline[linecolor=blue]{<-]}(5.25,1.0)(7.0,1.0)\rput{0}(6.125,0.8){$c_{1}$}
\psline[linecolor=red]{[->}(0.0,3.0)(1.75,3.0)\rput{0}(0.875,2.8){$z_{100}$}
\pscircle[linecolor=red,fillcolor=black,fillstyle=solid](0.0,3.0){0.075}
\pscircle[linecolor=red,fillcolor=black,fillstyle=solid](7.0,3.0){0.075}
\pscircle[linecolor=red,fillcolor=white,fillstyle=solid](1.75,3.0){0.075}
\pscircle[linecolor=red,fillcolor=white,fillstyle=solid](5.25,3.0){0.075}
\psline[linecolor=red]{<-]}(5.25,3.0)(7.0,3.0)\rput{0}(6.125,2.8){$z_{100}$}
\psline[linecolor=red]{[->}(1.75,4.0)(3.5,4.0)\rput{0}(2.625,3.8){$z_{101}$}
\pscircle[linecolor=red,fillcolor=black,fillstyle=solid](1.75,4.0){0.075}
\pscircle[linecolor=red,fillcolor=black,fillstyle=solid](5.25,5.0){0.075}
\pscircle[linecolor=red,fillcolor=white,fillstyle=solid](3.5,4.0){0.075}
\pscircle[linecolor=red,fillcolor=white,fillstyle=solid](3.5,5.0){0.075}
\psline[linecolor=red]{<-]}(3.5,5.0)(5.25,5.0)\rput{0}(4.375,4.8){$z_{101}$}
\end{pspicture}
\end{center}
{\bf GE Information}:  
Carrier: [0-1:z1-.] ;  
Carrier Dual: [2-3:z1+.] ;  
Critical Boundary: 1;  
The GE above is non-degenerate.  This GE is {\em not} a leaf in the GE tree.   It has 1 valid prints (descendents).  \\[0.1in]
   It has 1 legal carrier-to-dual prints, as follows:
\begin{verbatim}
     Print 1: =0=3*<1=2*
\end{verbatim}
We proceed.\\[0.2in]
\subsection*{Generalized Equation root-3.1}
\label{root-3.1}We begin from the GE root-3 (see pp. \pageref{root-3}).  {We consider its print}
\begin{verbatim}
     Print 1: =0=3*<1=2*
\end{verbatim}
{\bf Sequence of actions in performing the Print 1:}\\
{\underline{Step 1}:} Moved (old) base [0-1:z1-.]  to (new) boundaries 3 - 2.\\
{\underline{Step 2}:} Moved (old) base [0-1:z100+.]  to (new) boundaries 3 - 2.\\
{\underline{Step 3}:} Collapsed (new) base [2-3:z1+.]  to the empty base (3,3).
\\
{\underline{Step 4}:} Deleted (new) boundary 0 because it is not used inside any base.  This will cause renumbering of higher numbered boundaries.
\\[0.1in]
{Upon applying the print, the GE we obtain---which we refer to as root-3.1---is illustrated below:}
\begin{center}
\begin{pspicture}(-0.5,-0.5)(7.5,6.5)
\psline[linecolor=black]{-}(0.0,0.0)(0.0,6.0)\rput{0}(0.0,0.0){$0$}
\rput{0}(0.0,6.0){$0$}
\psline[linecolor=black]{-}(2.3333333333333335,0.0)(2.3333333333333335,6.0)\rput{0}(2.3333333333333335,0.0){$1$}
\rput{0}(2.3333333333333335,6.0){$1$}
\psline[linecolor=black]{-}(4.666666666666667,0.0)(4.666666666666667,6.0)\rput{0}(4.666666666666667,0.0){$2$}
\rput{0}(4.666666666666667,6.0){$2$}
\psline[linecolor=black]{-}(7.0,0.0)(7.0,6.0)\rput{0}(7.0,0.0){$3$}
\rput{0}(7.0,6.0){$3$}
\psline[linecolor=blue]{[->}(0.0,0.75)(2.3333333333333335,0.75)\rput{0}(1.1666666666666667,0.55){$c_{1}$}
\psline[linecolor=red]{[->}(4.666666666666667,0.75)(4.666666666666667,0.75)\rput{0}(4.666666666666667,0.55){$z_{1}$}
\psline[linecolor=red]{[->}(4.666666666666667,1.5)(4.666666666666667,1.5)\rput{0}(4.666666666666667,1.3){$z_{1}$}
\pscircle[linecolor=red,fillcolor=black,fillstyle=solid](4.666666666666667,1.5){0.075}
\pscircle[linecolor=red,fillcolor=black,fillstyle=solid](4.666666666666667,0.75){0.075}
\psline[linecolor=blue]{<-]}(4.666666666666667,2.25)(7.0,2.25)\rput{0}(5.833333333333334,2.05){$c_{1}$}
\psline[linecolor=red]{<-]}(2.3333333333333335,3.0)(4.666666666666667,3.0)\rput{0}(3.5,2.8){$z_{100}$}
\pscircle[linecolor=red,fillcolor=black,fillstyle=solid](2.3333333333333335,3.0){0.075}
\pscircle[linecolor=red,fillcolor=black,fillstyle=solid](4.666666666666667,3.75){0.075}
\pscircle[linecolor=red,fillcolor=white,fillstyle=solid](4.666666666666667,3.0){0.075}
\pscircle[linecolor=red,fillcolor=white,fillstyle=solid](7.0,3.75){0.075}
\psline[linecolor=red]{<-]}(4.666666666666667,3.75)(7.0,3.75)\rput{0}(5.833333333333334,3.55){$z_{100}$}
\psline[linecolor=red]{[->}(0.0,1.5)(2.3333333333333335,1.5)\rput{0}(1.1666666666666667,1.3){$z_{101}$}
\pscircle[linecolor=red,fillcolor=black,fillstyle=solid](0.0,1.5){0.075}
\pscircle[linecolor=red,fillcolor=black,fillstyle=solid](4.666666666666667,4.5){0.075}
\pscircle[linecolor=red,fillcolor=white,fillstyle=solid](2.3333333333333335,1.5){0.075}
\pscircle[linecolor=red,fillcolor=white,fillstyle=solid](2.3333333333333335,4.5){0.075}
\psline[linecolor=red]{<-]}(2.3333333333333335,4.5)(4.666666666666667,4.5)\rput{0}(3.5,4.3){$z_{101}$}
\end{pspicture}
\end{center}
{\bf GE Information}:  
Carrier: [0-1:z101+.] ;  
Carrier Dual: [1-2:z101-.] ;  
Critical Boundary: 1;  
The GE above is non-degenerate.  This GE is {\em not} a leaf in the GE tree.   It has 1 valid prints (descendents).  \\[0.1in]
   It has 1 legal carrier-to-dual prints, as follows:
\begin{verbatim}
     Print 1: =0=2*<1=1*
\end{verbatim}
This completes the consideration of root-3.1, as derived from the application of a print to root-3.\\[0.1in]
\subsection*{Generalized Equation root-3.1.1}
\label{root-3.1.1}We begin from the GE root-3.1 (see pp. \pageref{root-3.1}).  {We consider its print}
\begin{verbatim}
     Print 1: =0=2*<1=1*
\end{verbatim}
{\bf Sequence of actions in performing the Print 1:}\\
{\underline{Step 1}:} Moved (old) base [0-1:z101+.]  to (new) boundaries 2 - 1.\\
{\underline{Step 2}:} Moved (old) base [0-1:c1+.]  to (new) boundaries 2 - 1.\\
{\underline{Step 3}:} Collapsed (new) base [1-2:z101-.]  to the empty base (2,2).
\\
{\underline{Step 4}:} Deleted (new) boundary 0 because it is not used inside any base.  This will cause renumbering of higher numbered boundaries.
\\[0.1in]
{Upon applying the print, the GE we obtain---which we refer to as root-3.1.1---is illustrated below:}
\begin{center}
\begin{pspicture}(-0.5,-0.5)(7.5,6.5)
\psline[linecolor=black]{-}(0.0,0.0)(0.0,6.0)\rput{0}(0.0,0.0){$0$}
\rput{0}(0.0,6.0){$0$}
\psline[linecolor=black]{-}(3.5,0.0)(3.5,6.0)\rput{0}(3.5,0.0){$1$}
\rput{0}(3.5,6.0){$1$}
\psline[linecolor=black]{-}(7.0,0.0)(7.0,6.0)\rput{0}(7.0,0.0){$2$}
\rput{0}(7.0,6.0){$2$}
\psline[linecolor=blue]{<-]}(0.0,0.6)(3.5,0.6)\rput{0}(1.75,0.39999999999999997){$c_{1}$}
\psline[linecolor=red]{[->}(3.5,1.2)(3.5,1.2)\rput{0}(3.5,1.0){$z_{1}$}
\pscircle[linecolor=red,fillcolor=black,fillstyle=solid](3.5,1.2){0.075}
\pscircle[linecolor=red,fillcolor=black,fillstyle=solid](3.5,1.7999999999999998){0.075}
\psline[linecolor=red]{[->}(3.5,1.7999999999999998)(3.5,1.7999999999999998)\rput{0}(3.5,1.5999999999999999){$z_{1}$}
\psline[linecolor=blue]{<-]}(3.5,2.4)(7.0,2.4)\rput{0}(5.25,2.1999999999999997){$c_{1}$}
\psline[linecolor=red]{<-]}(0.0,3.0)(3.5,3.0)\rput{0}(1.75,2.8){$z_{100}$}
\pscircle[linecolor=red,fillcolor=black,fillstyle=solid](0.0,3.0){0.075}
\pscircle[linecolor=red,fillcolor=black,fillstyle=solid](3.5,3.5999999999999996){0.075}
\pscircle[linecolor=red,fillcolor=white,fillstyle=solid](3.5,3.0){0.075}
\pscircle[linecolor=red,fillcolor=white,fillstyle=solid](7.0,3.5999999999999996){0.075}
\psline[linecolor=red]{<-]}(3.5,3.5999999999999996)(7.0,3.5999999999999996)\rput{0}(5.25,3.3999999999999995){$z_{100}$}
\psline[linecolor=red]{<-]}(3.5,4.2)(3.5,4.2)\rput{0}(3.5,4.0){$z_{101}$}
\psline[linecolor=red]{<-]}(3.5,4.8)(3.5,4.8)\rput{0}(3.5,4.6){$z_{101}$}
\pscircle[linecolor=red,fillcolor=black,fillstyle=solid](3.5,4.8){0.075}
\pscircle[linecolor=red,fillcolor=black,fillstyle=solid](3.5,4.2){0.075}
\end{pspicture}
\end{center}
{\bf GE Information}:  
Carrier: [0-1:z100-.] ;  
Carrier Dual: [1-2:z100-.] ;  
Critical Boundary: 1;  
The GE above is non-degenerate.  This GE is {\em not} a leaf in the GE tree.   It has 1 valid prints (descendents).  \\[0.1in]
   It has 1 legal carrier-to-dual prints, as follows:
\begin{verbatim}
     Print 1: =0=1*<1=2*
\end{verbatim}
This completes the consideration of root-3.1.1, as derived from the application of a print to root-3.1.\\[0.1in]
\subsection*{Generalized Equation root-3.1.1.1}
\label{root-3.1.1.1}We begin from the GE root-3.1.1 (see pp. \pageref{root-3.1.1}).  {We consider its print}
\begin{verbatim}
     Print 1: =0=1*<1=2*
\end{verbatim}
{\bf Sequence of actions in performing the Print 1:}\\
{\underline{Step 1}:} Moved (old) base [0-1:z100-.]  to (new) boundaries 1 - 2.\\
{\underline{Step 2}:} Moved (old) base [0-1:c1-.]  to (new) boundaries 1 - 2.\\
{\underline{Step 3}:} Collapsed (new) base [1-2:z100-.]  to the empty base (2,2).
\\
{\underline{Step 4}:} Deleted (new) boundary 0 because it is not used inside any base.  This will cause renumbering of higher numbered boundaries.
\\[0.1in]
{Upon applying the print, the GE we obtain---which we refer to as root-3.1.1.1---is illustrated below:}
\begin{center}
\begin{pspicture}(-0.5,-0.5)(7.5,6.5)
\psline[linecolor=black]{-}(0.0,0.0)(0.0,6.0)\rput{0}(0.0,0.0){$0$}
\rput{0}(0.0,6.0){$0$}
\psline[linecolor=black]{-}(7.0,0.0)(7.0,6.0)\rput{0}(7.0,0.0){$1$}
\rput{0}(7.0,6.0){$1$}
\psline[linecolor=blue]{<-]}(0.0,0.75)(7.0,0.75)\rput{0}(3.5,0.55){$c_{1}$}
\psline[linecolor=red]{[->}(0.0,1.5)(0.0,1.5)\rput{0}(0.0,1.3){$z_{1}$}
\pscircle[linecolor=red,fillcolor=black,fillstyle=solid](0.0,1.5){0.075}
\pscircle[linecolor=red,fillcolor=black,fillstyle=solid](0.0,2.25){0.075}
\psline[linecolor=red]{[->}(0.0,2.25)(0.0,2.25)\rput{0}(0.0,2.05){$z_{1}$}
\psline[linecolor=blue]{<-]}(0.0,3.0)(7.0,3.0)\rput{0}(3.5,2.8){$c_{1}$}
\psline[linecolor=red]{<-]}(7.0,1.5)(7.0,1.5)\rput{0}(7.0,1.3){$z_{100}$}
\psline[linecolor=red]{<-]}(7.0,2.25)(7.0,2.25)\rput{0}(7.0,2.05){$z_{100}$}
\pscircle[linecolor=red,fillcolor=black,fillstyle=solid](7.0,2.25){0.075}
\pscircle[linecolor=red,fillcolor=black,fillstyle=solid](7.0,1.5){0.075}
\psline[linecolor=red]{<-]}(0.0,3.75)(0.0,3.75)\rput{0}(0.0,3.55){$z_{101}$}
\pscircle[linecolor=red,fillcolor=black,fillstyle=solid](0.0,3.75){0.075}
\pscircle[linecolor=red,fillcolor=black,fillstyle=solid](0.0,4.5){0.075}
\psline[linecolor=red]{<-]}(0.0,4.5)(0.0,4.5)\rput{0}(0.0,4.3){$z_{101}$}
\end{pspicture}
\end{center}
{\bf GE Information}:  
Carrier: [0-0:z1+.] ;  
Carrier Dual: [0-0:z1+.] ;  
Critical Boundary: 0;  
The GE above is non-degenerate.  This GE is a leaf in the GE tree.  We have effectively found a solution!\\[0.1in]
This completes the consideration of root-3.1.1.1, as derived from the application of a print to root-3.1.1.\\[0.1in]
\newpage
\section{Cancellation scheme \#$4$}
\begin{center}
\begin{pspicture}(-0.5,-0.5)(6.5,6.5)
{\psset{fillstyle=ccslope,slopebegin=yellow!40,slopeend=gray}
\cnodeput(0.00,0.00){0}{\strut\boldmath$0$}
\cnodeput(4.12,4.02){1}{\strut\boldmath$1$}
\cnodeput(6.00,6.00){2}{\strut\boldmath$2$}
\cnodeput(2.09,1.98){3}{\strut\boldmath$3$}
}
\newcommand\arc[3]{%
  \ncline{#1}{#2}{#3}
}
\arc{-}{1}{3}{}
\arc{-}{0}{3}{}
\arc{-}{1}{2}{}
\pscurve[linecolor=red]{<<-|}(4.29,3.85)(3.28,2.82)(2.27,1.80)(1.22,0.81)(0.17,-0.18)\rput{45}(3.28,2.82){$z_{1}$}
\psline[linecolor=blue]{|->>}(6.18,5.83)(5.24,4.84)(4.30,3.85)\rput{46}(5.24,4.84){$c_{1}$}
\pscurve[linecolor=red]{|->>}(1.91,2.15)(2.93,3.18)(3.94,4.20)(4.88,5.18)(5.82,6.17)\rput{45}(2.93,3.18){$z_{1}$}
\psline[linecolor=blue]{<<-|}(-0.17,0.18)(0.87,1.17)(1.92,2.16)\rput{43}(0.87,1.17){$c_{1}$}
\end{pspicture}
\end{center}
\begin{center}
\begin{tabular}{|ll|}
\hline
$z_{1}^{-1}$ & $1\leftarrow 3\leftarrow 0$\\
$c_{1}$ & $2\leftarrow 1$\\
$z_{1}$ & $3\leftarrow 1\leftarrow 2$\\
$c_{1}^{-1}$ & $0\leftarrow 3$\\
\hline
\end{tabular}
\end{center}
\subsection*{Generalized Equation root-4}
\label{root-4}Below is the root GE obtained from the cancellation diagram above.\begin{center}
\begin{pspicture}(-0.5,-0.5)(7.5,6.5)
\psline[linecolor=black]{-}(0.0,0.0)(0.0,6.0)\rput{0}(0.0,0.0){$0$}
\rput{0}(0.0,6.0){$0$}
\psline[linecolor=black]{-}(1.1666666666666667,0.0)(1.1666666666666667,6.0)\rput{0}(1.1666666666666667,0.0){$1$}
\rput{0}(1.1666666666666667,6.0){$1$}
\psline[linecolor=black]{-}(2.3333333333333335,0.0)(2.3333333333333335,6.0)\rput{0}(2.3333333333333335,0.0){$2$}
\rput{0}(2.3333333333333335,6.0){$2$}
\psline[linecolor=black]{-}(3.5,0.0)(3.5,6.0)\rput{0}(3.5,0.0){$3$}
\rput{0}(3.5,6.0){$3$}
\psline[linecolor=black]{-}(4.666666666666667,0.0)(4.666666666666667,6.0)\rput{0}(4.666666666666667,0.0){$4$}
\rput{0}(4.666666666666667,6.0){$4$}
\psline[linecolor=black]{-}(5.833333333333334,0.0)(5.833333333333334,6.0)\rput{0}(5.833333333333334,0.0){$5$}
\rput{0}(5.833333333333334,6.0){$5$}
\psline[linecolor=black]{-}(7.000000000000001,0.0)(7.000000000000001,6.0)\rput{0}(7.000000000000001,0.0){$6$}
\rput{0}(7.000000000000001,6.0){$6$}
\psline[linecolor=red]{<-]}(0.0,1.0)(2.3333333333333335,1.0)\rput{0}(1.1666666666666667,0.8){$z_{1}$}
\pscircle[linecolor=red,fillcolor=black,fillstyle=solid](0.0,1.0){0.075}
\pscircle[linecolor=red,fillcolor=black,fillstyle=solid](5.833333333333334,1.0){0.075}
\pscircle[linecolor=red,fillcolor=white,fillstyle=solid](2.3333333333333335,1.0){0.075}
\pscircle[linecolor=red,fillcolor=white,fillstyle=solid](3.5,1.0){0.075}
\psline[linecolor=red]{[->}(3.5,1.0)(5.833333333333334,1.0)\rput{0}(4.666666666666667,0.8){$z_{1}$}
\psline[linecolor=blue]{[->}(2.3333333333333335,2.0)(3.5,2.0)\rput{0}(2.916666666666667,1.8){$c_{1}$}
\psline[linecolor=blue]{<-]}(5.833333333333334,2.0)(7.0,2.0)\rput{0}(6.416666666666667,1.8){$c_{1}$}
\psline[linecolor=red]{[->}(1.1666666666666667,3.0)(2.3333333333333335,3.0)\rput{0}(1.75,2.8){$z_{100}$}
\pscircle[linecolor=red,fillcolor=black,fillstyle=solid](1.1666666666666667,3.0){0.075}
\pscircle[linecolor=red,fillcolor=black,fillstyle=solid](5.833333333333334,3.0){0.075}
\pscircle[linecolor=red,fillcolor=white,fillstyle=solid](2.3333333333333335,3.0){0.075}
\pscircle[linecolor=red,fillcolor=white,fillstyle=solid](4.666666666666667,3.0){0.075}
\psline[linecolor=red]{<-]}(4.666666666666667,3.0)(5.833333333333334,3.0)\rput{0}(5.25,2.8){$z_{100}$}
\psline[linecolor=red]{[->}(0.0,2.0)(1.1666666666666667,2.0)\rput{0}(0.5833333333333334,1.8){$z_{101}$}
\pscircle[linecolor=red,fillcolor=black,fillstyle=solid](0.0,2.0){0.075}
\pscircle[linecolor=red,fillcolor=black,fillstyle=solid](7.0,4.0){0.075}
\pscircle[linecolor=red,fillcolor=white,fillstyle=solid](1.1666666666666667,2.0){0.075}
\pscircle[linecolor=red,fillcolor=white,fillstyle=solid](5.833333333333334,4.0){0.075}
\psline[linecolor=red]{<-]}(5.833333333333334,4.0)(7.0,4.0)\rput{0}(6.416666666666667,3.8){$z_{101}$}
\psline[linecolor=red]{[->}(2.3333333333333335,4.0)(3.5,4.0)\rput{0}(2.916666666666667,3.8){$z_{102}$}
\pscircle[linecolor=red,fillcolor=black,fillstyle=solid](2.3333333333333335,4.0){0.075}
\pscircle[linecolor=red,fillcolor=black,fillstyle=solid](4.666666666666667,5.0){0.075}
\pscircle[linecolor=red,fillcolor=white,fillstyle=solid](3.5,4.0){0.075}
\pscircle[linecolor=red,fillcolor=white,fillstyle=solid](3.5,5.0){0.075}
\psline[linecolor=red]{<-]}(3.5,5.0)(4.666666666666667,5.0)\rput{0}(4.083333333333334,4.8){$z_{102}$}
\end{pspicture}
\end{center}
{\bf GE Information}:  
Carrier: [0-2:z1-.] ;  
Carrier Dual: [3-5:z1+.] ;  
Critical Boundary: 2;  
The GE above is non-degenerate.  This GE is {\em not} a leaf in the GE tree.   It has 3 valid prints (descendents).  \\[0.1in]
   It has 3 legal carrier-to-dual prints, as follows:
\begin{verbatim}
     Print 1: =0=5*<1=4*<2=3*
     Print 2: =0=5*<1<4*<2=3*
     Print 3: =0=5*<4*<1<2=3*
\end{verbatim}
We proceed.\\[0.2in]
\subsection*{Generalized Equation root-4.1}
\label{root-4.1}We begin from the GE root-4 (see pp. \pageref{root-4}).  {We consider its print}
\begin{verbatim}
     Print 1: =0=5*<1=4*<2=3*
\end{verbatim}
{\bf Sequence of actions in performing the Print 1:}\\
{\underline{Step 1}:} Moved (old) base [0-2:z1-.]  to (new) boundaries 5 - 3.\\
{\underline{Step 2}:} Moved (old) base [1-2:z100+.]  to (new) boundaries 4 - 3.\\
{\underline{Step 3}:} Moved (old) base [0-1:z101+.]  to (new) boundaries 5 - 4.\\
{\underline{Step 4}:} Collapsed (new) base [3-5:z1+.]  to the empty base (5,5).
\\
{\underline{Step 5}:} Deleted (new) boundary 0 because it is not used inside any base.  This will cause renumbering of higher numbered boundaries.
\\
{\underline{Step 6}:} Deleted (new) boundary 1 because it is not used inside any base.  This will cause renumbering of higher numbered boundaries.
\\[0.1in]
{Upon applying the print, the GE we obtain---which we refer to as root-4.1---is illustrated below:}
\begin{center}
\begin{pspicture}(-0.5,-0.5)(7.5,6.5)
\psline[linecolor=black]{-}(0.0,0.0)(0.0,6.0)\rput{0}(0.0,0.0){$0$}
\rput{0}(0.0,6.0){$0$}
\psline[linecolor=black]{-}(1.75,0.0)(1.75,6.0)\rput{0}(1.75,0.0){$1$}
\rput{0}(1.75,6.0){$1$}
\psline[linecolor=black]{-}(3.5,0.0)(3.5,6.0)\rput{0}(3.5,0.0){$2$}
\rput{0}(3.5,6.0){$2$}
\psline[linecolor=black]{-}(5.25,0.0)(5.25,6.0)\rput{0}(5.25,0.0){$3$}
\rput{0}(5.25,6.0){$3$}
\psline[linecolor=black]{-}(7.0,0.0)(7.0,6.0)\rput{0}(7.0,0.0){$4$}
\rput{0}(7.0,6.0){$4$}
\psline[linecolor=red]{[->}(5.25,0.75)(5.25,0.75)\rput{0}(5.25,0.55){$z_{1}$}
\pscircle[linecolor=red,fillcolor=black,fillstyle=solid](5.25,0.75){0.075}
\pscircle[linecolor=red,fillcolor=black,fillstyle=solid](5.25,1.5){0.075}
\psline[linecolor=red]{[->}(5.25,1.5)(5.25,1.5)\rput{0}(5.25,1.3){$z_{1}$}
\psline[linecolor=blue]{[->}(0.0,0.75)(1.75,0.75)\rput{0}(0.875,0.55){$c_{1}$}
\psline[linecolor=blue]{<-]}(5.25,2.25)(7.0,2.25)\rput{0}(6.125,2.05){$c_{1}$}
\psline[linecolor=red]{<-]}(1.75,1.5)(3.5,1.5)\rput{0}(2.625,1.3){$z_{100}$}
\pscircle[linecolor=red,fillcolor=black,fillstyle=solid](1.75,1.5){0.075}
\pscircle[linecolor=red,fillcolor=black,fillstyle=solid](3.5,3.0){0.075}
\pscircle[linecolor=red,fillcolor=white,fillstyle=solid](3.5,1.5){0.075}
\pscircle[linecolor=red,fillcolor=white,fillstyle=solid](5.25,3.0){0.075}
\psline[linecolor=red]{<-]}(3.5,3.0)(5.25,3.0)\rput{0}(4.375,2.8){$z_{100}$}
\psline[linecolor=red]{<-]}(3.5,3.75)(5.25,3.75)\rput{0}(4.375,3.55){$z_{101}$}
\pscircle[linecolor=red,fillcolor=black,fillstyle=solid](3.5,3.75){0.075}
\pscircle[linecolor=red,fillcolor=black,fillstyle=solid](5.25,4.5){0.075}
\pscircle[linecolor=red,fillcolor=white,fillstyle=solid](5.25,3.75){0.075}
\pscircle[linecolor=red,fillcolor=white,fillstyle=solid](7.0,4.5){0.075}
\psline[linecolor=red]{<-]}(5.25,4.5)(7.0,4.5)\rput{0}(6.125,4.3){$z_{101}$}
\psline[linecolor=red]{[->}(0.0,2.25)(1.75,2.25)\rput{0}(0.875,2.05){$z_{102}$}
\pscircle[linecolor=red,fillcolor=black,fillstyle=solid](0.0,2.25){0.075}
\pscircle[linecolor=red,fillcolor=black,fillstyle=solid](3.5,4.5){0.075}
\pscircle[linecolor=red,fillcolor=white,fillstyle=solid](1.75,2.25){0.075}
\pscircle[linecolor=red,fillcolor=white,fillstyle=solid](1.75,4.5){0.075}
\psline[linecolor=red]{<-]}(1.75,4.5)(3.5,4.5)\rput{0}(2.625,4.3){$z_{102}$}
\end{pspicture}
\end{center}
{\bf GE Information}:  
Carrier: [0-1:z102+.] ;  
Carrier Dual: [1-2:z102-.] ;  
Critical Boundary: 1;  
The GE above is non-degenerate.  This GE is {\em not} a leaf in the GE tree.   It has 1 valid prints (descendents).  \\[0.1in]
   It has 1 legal carrier-to-dual prints, as follows:
\begin{verbatim}
     Print 1: =0=2*<1=1*
\end{verbatim}
This completes the consideration of root-4.1, as derived from the application of a print to root-4.\\[0.1in]
\subsection*{Generalized Equation root-4.2}
\label{root-4.2}We begin from the GE root-4 (see pp. \pageref{root-4}).  {We consider its print}
\begin{verbatim}
     Print 2: =0=5*<1<4*<2=3*
\end{verbatim}
{\bf Sequence of actions in performing the Print 2:}\\
{\underline{Step 1}:} Added (new) boundary 5.\\
{\underline{Step 2}:} Moved (old) base [0-2:z1-.]  to (new) boundaries 6 - 3.\\
{\underline{Step 3}:} Moved (old) base [1-2:z100+.]  to (new) boundaries 5 - 3.\\
{\underline{Step 4}:} Moved (old) base [0-1:z101+.]  to (new) boundaries 6 - 5.\\
{\underline{Step 5}:} Collapsed (new) base [3-6:z1+.]  to the empty base (6,6).
\\
{\underline{Step 6}:} Deleted (new) boundary 0 because it is not used inside any base.  This will cause renumbering of higher numbered boundaries.
\\
{\underline{Step 7}:} Deleted (new) boundary 1 because it is not used inside any base.  This will cause renumbering of higher numbered boundaries.
\\[0.1in]
{Upon applying the print, the GE we obtain---which we refer to as root-4.2---is illustrated below:}
\begin{center}
\begin{pspicture}(-0.5,-0.5)(7.5,6.5)
\psline[linecolor=black]{-}(0.0,0.0)(0.0,6.0)\rput{0}(0.0,0.0){$0$}
\rput{0}(0.0,6.0){$0$}
\psline[linecolor=black]{-}(1.4,0.0)(1.4,6.0)\rput{0}(1.4,0.0){$1$}
\rput{0}(1.4,6.0){$1$}
\psline[linecolor=black]{-}(2.8,0.0)(2.8,6.0)\rput{0}(2.8,0.0){$2$}
\rput{0}(2.8,6.0){$2$}
\psline[linecolor=black]{-}(4.199999999999999,0.0)(4.199999999999999,6.0)\rput{0}(4.199999999999999,0.0){$3$}
\rput{0}(4.199999999999999,6.0){$3$}
\psline[linecolor=black]{-}(5.6,0.0)(5.6,6.0)\rput{0}(5.6,0.0){$4$}
\rput{0}(5.6,6.0){$4$}
\psline[linecolor=black]{-}(7.0,0.0)(7.0,6.0)\rput{0}(7.0,0.0){$5$}
\rput{0}(7.0,6.0){$5$}
\psline[linecolor=red]{[->}(5.6,0.75)(5.6,0.75)\rput{0}(5.6,0.55){$z_{1}$}
\pscircle[linecolor=red,fillcolor=black,fillstyle=solid](5.6,0.75){0.075}
\pscircle[linecolor=red,fillcolor=black,fillstyle=solid](5.6,1.5){0.075}
\psline[linecolor=red]{[->}(5.6,1.5)(5.6,1.5)\rput{0}(5.6,1.3){$z_{1}$}
\psline[linecolor=blue]{[->}(0.0,0.75)(1.4,0.75)\rput{0}(0.7,0.55){$c_{1}$}
\psline[linecolor=blue]{<-]}(5.6,2.25)(7.0,2.25)\rput{0}(6.3,2.05){$c_{1}$}
\psline[linecolor=red]{<-]}(1.4,1.5)(4.199999999999999,1.5)\rput{0}(2.8,1.3){$z_{100}$}
\pscircle[linecolor=red,fillcolor=black,fillstyle=solid](1.4,1.5){0.075}
\pscircle[linecolor=red,fillcolor=black,fillstyle=solid](2.8,3.0){0.075}
\pscircle[linecolor=red,fillcolor=white,fillstyle=solid](4.199999999999999,1.5){0.075}
\pscircle[linecolor=red,fillcolor=white,fillstyle=solid](5.6,3.0){0.075}
\psline[linecolor=red]{<-]}(2.8,3.0)(5.6,3.0)\rput{0}(4.199999999999999,2.8){$z_{100}$}
\psline[linecolor=red]{<-]}(4.199999999999999,3.75)(5.6,3.75)\rput{0}(4.8999999999999995,3.55){$z_{101}$}
\pscircle[linecolor=red,fillcolor=black,fillstyle=solid](4.199999999999999,3.75){0.075}
\pscircle[linecolor=red,fillcolor=black,fillstyle=solid](5.6,4.5){0.075}
\pscircle[linecolor=red,fillcolor=white,fillstyle=solid](5.6,3.75){0.075}
\pscircle[linecolor=red,fillcolor=white,fillstyle=solid](7.0,4.5){0.075}
\psline[linecolor=red]{<-]}(5.6,4.5)(7.0,4.5)\rput{0}(6.3,4.3){$z_{101}$}
\psline[linecolor=red]{[->}(0.0,2.25)(1.4,2.25)\rput{0}(0.7,2.05){$z_{102}$}
\pscircle[linecolor=red,fillcolor=black,fillstyle=solid](0.0,2.25){0.075}
\pscircle[linecolor=red,fillcolor=black,fillstyle=solid](2.8,3.75){0.075}
\pscircle[linecolor=red,fillcolor=white,fillstyle=solid](1.4,2.25){0.075}
\pscircle[linecolor=red,fillcolor=white,fillstyle=solid](1.4,3.75){0.075}
\psline[linecolor=red]{<-]}(1.4,3.75)(2.8,3.75)\rput{0}(2.0999999999999996,3.55){$z_{102}$}
\end{pspicture}
\end{center}
{\bf GE Information}:  
Carrier: [0-1:z102+.] ;  
Carrier Dual: [1-2:z102-.] ;  
Critical Boundary: 1;  
The GE above is non-degenerate.  This GE is {\em not} a leaf in the GE tree.   It has 1 valid prints (descendents).  \\[0.1in]
   It has 1 legal carrier-to-dual prints, as follows:
\begin{verbatim}
     Print 1: =0=2*<1=1*
\end{verbatim}
This completes the consideration of root-4.2, as derived from the application of a print to root-4.\\[0.1in]
\subsection*{Generalized Equation root-4.3}
\label{root-4.3}We begin from the GE root-4 (see pp. \pageref{root-4}).  {We consider its print}
\begin{verbatim}
     Print 3: =0=5*<4*<1<2=3*
\end{verbatim}
{\bf Sequence of actions in performing the Print 3:}\\
{\underline{Step 1}:} Added (new) boundary 4.\\
{\underline{Step 2}:} Moved (old) base [0-2:z1-.]  to (new) boundaries 6 - 3.\\
{\underline{Step 3}:} Moved (old) base [1-2:z100+.]  to (new) boundaries 4 - 3.\\
{\underline{Step 4}:} Moved (old) base [0-1:z101+.]  to (new) boundaries 6 - 4.\\
{\underline{Step 5}:} Collapsed (new) base [3-6:z1+.]  to the empty base (6,6).
\\
{\underline{Step 6}:} Deleted (new) boundary 0 because it is not used inside any base.  This will cause renumbering of higher numbered boundaries.
\\
{\underline{Step 7}:} Deleted (new) boundary 1 because it is not used inside any base.  This will cause renumbering of higher numbered boundaries.
\\[0.1in]
{Upon applying the print, the GE we obtain---which we refer to as root-4.3---is illustrated below:}
\begin{center}
\begin{pspicture}(-0.5,-0.5)(7.5,6.5)
\psline[linecolor=black]{-}(0.0,0.0)(0.0,6.0)\rput{0}(0.0,0.0){$0$}
\rput{0}(0.0,6.0){$0$}
\psline[linecolor=black]{-}(1.4,0.0)(1.4,6.0)\rput{0}(1.4,0.0){$1$}
\rput{0}(1.4,6.0){$1$}
\psline[linecolor=black]{-}(2.8,0.0)(2.8,6.0)\rput{0}(2.8,0.0){$2$}
\rput{0}(2.8,6.0){$2$}
\psline[linecolor=black]{-}(4.199999999999999,0.0)(4.199999999999999,6.0)\rput{0}(4.199999999999999,0.0){$3$}
\rput{0}(4.199999999999999,6.0){$3$}
\psline[linecolor=black]{-}(5.6,0.0)(5.6,6.0)\rput{0}(5.6,0.0){$4$}
\rput{0}(5.6,6.0){$4$}
\psline[linecolor=black]{-}(7.0,0.0)(7.0,6.0)\rput{0}(7.0,0.0){$5$}
\rput{0}(7.0,6.0){$5$}
\psline[linecolor=red]{[->}(5.6,0.75)(5.6,0.75)\rput{0}(5.6,0.55){$z_{1}$}
\pscircle[linecolor=red,fillcolor=black,fillstyle=solid](5.6,0.75){0.075}
\pscircle[linecolor=red,fillcolor=black,fillstyle=solid](5.6,1.5){0.075}
\psline[linecolor=red]{[->}(5.6,1.5)(5.6,1.5)\rput{0}(5.6,1.3){$z_{1}$}
\psline[linecolor=blue]{[->}(0.0,0.75)(1.4,0.75)\rput{0}(0.7,0.55){$c_{1}$}
\psline[linecolor=blue]{<-]}(5.6,2.25)(7.0,2.25)\rput{0}(6.3,2.05){$c_{1}$}
\psline[linecolor=red]{<-]}(1.4,1.5)(2.8,1.5)\rput{0}(2.0999999999999996,1.3){$z_{100}$}
\pscircle[linecolor=red,fillcolor=black,fillstyle=solid](1.4,1.5){0.075}
\pscircle[linecolor=red,fillcolor=black,fillstyle=solid](4.199999999999999,3.0){0.075}
\pscircle[linecolor=red,fillcolor=white,fillstyle=solid](2.8,1.5){0.075}
\pscircle[linecolor=red,fillcolor=white,fillstyle=solid](5.6,3.0){0.075}
\psline[linecolor=red]{<-]}(4.199999999999999,3.0)(5.6,3.0)\rput{0}(4.8999999999999995,2.8){$z_{100}$}
\psline[linecolor=red]{<-]}(2.8,3.75)(5.6,3.75)\rput{0}(4.199999999999999,3.55){$z_{101}$}
\pscircle[linecolor=red,fillcolor=black,fillstyle=solid](2.8,3.75){0.075}
\pscircle[linecolor=red,fillcolor=black,fillstyle=solid](5.6,4.5){0.075}
\pscircle[linecolor=red,fillcolor=white,fillstyle=solid](5.6,3.75){0.075}
\pscircle[linecolor=red,fillcolor=white,fillstyle=solid](7.0,4.5){0.075}
\psline[linecolor=red]{<-]}(5.6,4.5)(7.0,4.5)\rput{0}(6.3,4.3){$z_{101}$}
\psline[linecolor=red]{[->}(0.0,2.25)(1.4,2.25)\rput{0}(0.7,2.05){$z_{102}$}
\pscircle[linecolor=red,fillcolor=black,fillstyle=solid](0.0,2.25){0.075}
\pscircle[linecolor=red,fillcolor=black,fillstyle=solid](4.199999999999999,4.5){0.075}
\pscircle[linecolor=red,fillcolor=white,fillstyle=solid](1.4,2.25){0.075}
\pscircle[linecolor=red,fillcolor=white,fillstyle=solid](1.4,4.5){0.075}
\psline[linecolor=red]{<-]}(1.4,4.5)(4.199999999999999,4.5)\rput{0}(2.8,4.3){$z_{102}$}
\end{pspicture}
\end{center}
{\bf GE Information}:  
Carrier: [0-1:z102+.] ;  
Carrier Dual: [1-3:z102-.] ;  
Critical Boundary: 1;  
Observe the following facts about this GE:
The base [0-1:z102+.]  has constraints with its dual that stretch the constant segment 0 - 1 to length different from 1.  The base [4-5:z101-.]  has constraints with its dual that stretch the constant segment 4 - 5 to length different from 1.  These observations show that the GE above is degenerate.  This GE is a leaf in the GE tree.  This branch of the tree has led us to a dead end.\\[0.1in]
This completes the consideration of root-4.3, as derived from the application of a print to root-4.\\[0.1in]
\subsection*{Generalized Equation root-4.1.1}
\label{root-4.1.1}We begin from the GE root-4.1 (see pp. \pageref{root-4.1}).  {We consider its print}
\begin{verbatim}
     Print 1: =0=2*<1=1*
\end{verbatim}
{\bf Sequence of actions in performing the Print 1:}\\
{\underline{Step 1}:} Moved (old) base [0-1:z102+.]  to (new) boundaries 2 - 1.\\
{\underline{Step 2}:} Moved (old) base [0-1:c1+.]  to (new) boundaries 2 - 1.\\
{\underline{Step 3}:} Collapsed (new) base [1-2:z102-.]  to the empty base (2,2).
\\
{\underline{Step 4}:} Deleted (new) boundary 0 because it is not used inside any base.  This will cause renumbering of higher numbered boundaries.
\\[0.1in]
{Upon applying the print, the GE we obtain---which we refer to as root-4.1.1---is illustrated below:}
\begin{center}
\begin{pspicture}(-0.5,-0.5)(7.5,6.5)
\psline[linecolor=black]{-}(0.0,0.0)(0.0,6.0)\rput{0}(0.0,0.0){$0$}
\rput{0}(0.0,6.0){$0$}
\psline[linecolor=black]{-}(2.3333333333333335,0.0)(2.3333333333333335,6.0)\rput{0}(2.3333333333333335,0.0){$1$}
\rput{0}(2.3333333333333335,6.0){$1$}
\psline[linecolor=black]{-}(4.666666666666667,0.0)(4.666666666666667,6.0)\rput{0}(4.666666666666667,0.0){$2$}
\rput{0}(4.666666666666667,6.0){$2$}
\psline[linecolor=black]{-}(7.0,0.0)(7.0,6.0)\rput{0}(7.0,0.0){$3$}
\rput{0}(7.0,6.0){$3$}
\psline[linecolor=red]{[->}(4.666666666666667,0.75)(4.666666666666667,0.75)\rput{0}(4.666666666666667,0.55){$z_{1}$}
\psline[linecolor=red]{[->}(4.666666666666667,1.5)(4.666666666666667,1.5)\rput{0}(4.666666666666667,1.3){$z_{1}$}
\pscircle[linecolor=red,fillcolor=black,fillstyle=solid](4.666666666666667,1.5){0.075}
\pscircle[linecolor=red,fillcolor=black,fillstyle=solid](4.666666666666667,0.75){0.075}
\psline[linecolor=blue]{<-]}(0.0,0.75)(2.3333333333333335,0.75)\rput{0}(1.1666666666666667,0.55){$c_{1}$}
\psline[linecolor=blue]{<-]}(4.666666666666667,2.25)(7.0,2.25)\rput{0}(5.833333333333334,2.05){$c_{1}$}
\psline[linecolor=red]{<-]}(0.0,1.5)(2.3333333333333335,1.5)\rput{0}(1.1666666666666667,1.3){$z_{100}$}
\pscircle[linecolor=red,fillcolor=black,fillstyle=solid](0.0,1.5){0.075}
\pscircle[linecolor=red,fillcolor=black,fillstyle=solid](2.3333333333333335,3.0){0.075}
\pscircle[linecolor=red,fillcolor=white,fillstyle=solid](2.3333333333333335,1.5){0.075}
\pscircle[linecolor=red,fillcolor=white,fillstyle=solid](4.666666666666667,3.0){0.075}
\psline[linecolor=red]{<-]}(2.3333333333333335,3.0)(4.666666666666667,3.0)\rput{0}(3.5,2.8){$z_{100}$}
\psline[linecolor=red]{<-]}(2.3333333333333335,3.75)(4.666666666666667,3.75)\rput{0}(3.5,3.55){$z_{101}$}
\pscircle[linecolor=red,fillcolor=black,fillstyle=solid](2.3333333333333335,3.75){0.075}
\pscircle[linecolor=red,fillcolor=black,fillstyle=solid](4.666666666666667,4.5){0.075}
\pscircle[linecolor=red,fillcolor=white,fillstyle=solid](4.666666666666667,3.75){0.075}
\pscircle[linecolor=red,fillcolor=white,fillstyle=solid](7.0,4.5){0.075}
\psline[linecolor=red]{<-]}(4.666666666666667,4.5)(7.0,4.5)\rput{0}(5.833333333333334,4.3){$z_{101}$}
\psline[linecolor=red]{<-]}(2.3333333333333335,2.25)(2.3333333333333335,2.25)\rput{0}(2.3333333333333335,2.05){$z_{102}$}
\pscircle[linecolor=red,fillcolor=black,fillstyle=solid](2.3333333333333335,2.25){0.075}
\pscircle[linecolor=red,fillcolor=black,fillstyle=solid](2.3333333333333335,4.5){0.075}
\psline[linecolor=red]{<-]}(2.3333333333333335,4.5)(2.3333333333333335,4.5)\rput{0}(2.3333333333333335,4.3){$z_{102}$}
\end{pspicture}
\end{center}
{\bf GE Information}:  
Carrier: [0-1:z100-.] ;  
Carrier Dual: [1-2:z100-.] ;  
Critical Boundary: 1;  
The GE above is non-degenerate.  This GE is {\em not} a leaf in the GE tree.   It has 1 valid prints (descendents).  \\[0.1in]
   It has 1 legal carrier-to-dual prints, as follows:
\begin{verbatim}
     Print 1: =0=1*<1=2*
\end{verbatim}
This completes the consideration of root-4.1.1, as derived from the application of a print to root-4.1.\\[0.1in]
\subsection*{Generalized Equation root-4.2.1}
\label{root-4.2.1}We begin from the GE root-4.2 (see pp. \pageref{root-4.2}).  {We consider its print}
\begin{verbatim}
     Print 1: =0=2*<1=1*
\end{verbatim}
{\bf Sequence of actions in performing the Print 1:}\\
{\underline{Step 1}:} Moved (old) base [0-1:z102+.]  to (new) boundaries 2 - 1.\\
{\underline{Step 2}:} Moved (old) base [0-1:c1+.]  to (new) boundaries 2 - 1.\\
{\underline{Step 3}:} Collapsed (new) base [1-2:z102-.]  to the empty base (2,2).
\\
{\underline{Step 4}:} Deleted (new) boundary 0 because it is not used inside any base.  This will cause renumbering of higher numbered boundaries.
\\[0.1in]
{Upon applying the print, the GE we obtain---which we refer to as root-4.2.1---is illustrated below:}
\begin{center}
\begin{pspicture}(-0.5,-0.5)(7.5,6.5)
\psline[linecolor=black]{-}(0.0,0.0)(0.0,6.0)\rput{0}(0.0,0.0){$0$}
\rput{0}(0.0,6.0){$0$}
\psline[linecolor=black]{-}(1.75,0.0)(1.75,6.0)\rput{0}(1.75,0.0){$1$}
\rput{0}(1.75,6.0){$1$}
\psline[linecolor=black]{-}(3.5,0.0)(3.5,6.0)\rput{0}(3.5,0.0){$2$}
\rput{0}(3.5,6.0){$2$}
\psline[linecolor=black]{-}(5.25,0.0)(5.25,6.0)\rput{0}(5.25,0.0){$3$}
\rput{0}(5.25,6.0){$3$}
\psline[linecolor=black]{-}(7.0,0.0)(7.0,6.0)\rput{0}(7.0,0.0){$4$}
\rput{0}(7.0,6.0){$4$}
\psline[linecolor=red]{[->}(5.25,0.75)(5.25,0.75)\rput{0}(5.25,0.55){$z_{1}$}
\pscircle[linecolor=red,fillcolor=black,fillstyle=solid](5.25,0.75){0.075}
\pscircle[linecolor=red,fillcolor=black,fillstyle=solid](5.25,1.5){0.075}
\psline[linecolor=red]{[->}(5.25,1.5)(5.25,1.5)\rput{0}(5.25,1.3){$z_{1}$}
\psline[linecolor=blue]{<-]}(0.0,0.75)(1.75,0.75)\rput{0}(0.875,0.55){$c_{1}$}
\psline[linecolor=blue]{<-]}(5.25,2.25)(7.0,2.25)\rput{0}(6.125,2.05){$c_{1}$}
\psline[linecolor=red]{<-]}(0.0,1.5)(3.5,1.5)\rput{0}(1.75,1.3){$z_{100}$}
\pscircle[linecolor=red,fillcolor=black,fillstyle=solid](0.0,1.5){0.075}
\pscircle[linecolor=red,fillcolor=black,fillstyle=solid](1.75,3.0){0.075}
\pscircle[linecolor=red,fillcolor=white,fillstyle=solid](3.5,1.5){0.075}
\pscircle[linecolor=red,fillcolor=white,fillstyle=solid](5.25,3.0){0.075}
\psline[linecolor=red]{<-]}(1.75,3.0)(5.25,3.0)\rput{0}(3.5,2.8){$z_{100}$}
\psline[linecolor=red]{<-]}(3.5,3.75)(5.25,3.75)\rput{0}(4.375,3.55){$z_{101}$}
\pscircle[linecolor=red,fillcolor=black,fillstyle=solid](3.5,3.75){0.075}
\pscircle[linecolor=red,fillcolor=black,fillstyle=solid](5.25,4.5){0.075}
\pscircle[linecolor=red,fillcolor=white,fillstyle=solid](5.25,3.75){0.075}
\pscircle[linecolor=red,fillcolor=white,fillstyle=solid](7.0,4.5){0.075}
\psline[linecolor=red]{<-]}(5.25,4.5)(7.0,4.5)\rput{0}(6.125,4.3){$z_{101}$}
\psline[linecolor=red]{<-]}(1.75,2.25)(1.75,2.25)\rput{0}(1.75,2.05){$z_{102}$}
\psline[linecolor=red]{<-]}(1.75,3.75)(1.75,3.75)\rput{0}(1.75,3.55){$z_{102}$}
\pscircle[linecolor=red,fillcolor=black,fillstyle=solid](1.75,3.75){0.075}
\pscircle[linecolor=red,fillcolor=black,fillstyle=solid](1.75,2.25){0.075}
\end{pspicture}
\end{center}
{\bf GE Information}:  
Carrier: [0-2:z100-.] ;  
Carrier Dual: [1-3:z100-.] ;  
Critical Boundary: 1;  
The GE above is non-degenerate.  This GE is {\em not} a leaf in the GE tree.   It has 2 valid prints (descendents).  \\[0.1in]
   It has 2 legal carrier-to-dual prints, as follows:
\begin{verbatim}
     Print 1: =0=1*<1=2*<2=3*
     Print 2: =0=1*<1<2*<2=3*
\end{verbatim}
This completes the consideration of root-4.2.1, as derived from the application of a print to root-4.2.\\[0.1in]
\subsection*{Generalized Equation root-4.1.1.1}
\label{root-4.1.1.1}We begin from the GE root-4.1.1 (see pp. \pageref{root-4.1.1}).  {We consider its print}
\begin{verbatim}
     Print 1: =0=1*<1=2*
\end{verbatim}
{\bf Sequence of actions in performing the Print 1:}\\
{\underline{Step 1}:} Moved (old) base [0-1:z100-.]  to (new) boundaries 1 - 2.\\
{\underline{Step 2}:} Moved (old) base [0-1:c1-.]  to (new) boundaries 1 - 2.\\
{\underline{Step 3}:} Collapsed (new) base [1-2:z100-.]  to the empty base (2,2).
\\
{\underline{Step 4}:} Deleted (new) boundary 0 because it is not used inside any base.  This will cause renumbering of higher numbered boundaries.
\\[0.1in]
{Upon applying the print, the GE we obtain---which we refer to as root-4.1.1.1---is illustrated below:}
\begin{center}
\begin{pspicture}(-0.5,-0.5)(7.5,6.5)
\psline[linecolor=black]{-}(0.0,0.0)(0.0,6.0)\rput{0}(0.0,0.0){$0$}
\rput{0}(0.0,6.0){$0$}
\psline[linecolor=black]{-}(3.5,0.0)(3.5,6.0)\rput{0}(3.5,0.0){$1$}
\rput{0}(3.5,6.0){$1$}
\psline[linecolor=black]{-}(7.0,0.0)(7.0,6.0)\rput{0}(7.0,0.0){$2$}
\rput{0}(7.0,6.0){$2$}
\psline[linecolor=red]{[->}(3.5,0.6)(3.5,0.6)\rput{0}(3.5,0.39999999999999997){$z_{1}$}
\pscircle[linecolor=red,fillcolor=black,fillstyle=solid](3.5,0.6){0.075}
\pscircle[linecolor=red,fillcolor=black,fillstyle=solid](3.5,1.2){0.075}
\psline[linecolor=red]{[->}(3.5,1.2)(3.5,1.2)\rput{0}(3.5,1.0){$z_{1}$}
\psline[linecolor=blue]{<-]}(0.0,1.7999999999999998)(3.5,1.7999999999999998)\rput{0}(1.75,1.5999999999999999){$c_{1}$}
\psline[linecolor=blue]{<-]}(3.5,2.4)(7.0,2.4)\rput{0}(5.25,2.1999999999999997){$c_{1}$}
\psline[linecolor=red]{<-]}(3.5,3.0)(3.5,3.0)\rput{0}(3.5,2.8){$z_{100}$}
\psline[linecolor=red]{<-]}(3.5,3.5999999999999996)(3.5,3.5999999999999996)\rput{0}(3.5,3.3999999999999995){$z_{100}$}
\pscircle[linecolor=red,fillcolor=black,fillstyle=solid](3.5,3.5999999999999996){0.075}
\pscircle[linecolor=red,fillcolor=black,fillstyle=solid](3.5,3.0){0.075}
\psline[linecolor=red]{<-]}(0.0,4.2)(3.5,4.2)\rput{0}(1.75,4.0){$z_{101}$}
\pscircle[linecolor=red,fillcolor=black,fillstyle=solid](0.0,4.2){0.075}
\pscircle[linecolor=red,fillcolor=black,fillstyle=solid](3.5,4.8){0.075}
\pscircle[linecolor=red,fillcolor=white,fillstyle=solid](3.5,4.2){0.075}
\pscircle[linecolor=red,fillcolor=white,fillstyle=solid](7.0,4.8){0.075}
\psline[linecolor=red]{<-]}(3.5,4.8)(7.0,4.8)\rput{0}(5.25,4.6){$z_{101}$}
\psline[linecolor=red]{<-]}(0.0,0.6)(0.0,0.6)\rput{0}(0.0,0.39999999999999997){$z_{102}$}
\psline[linecolor=red]{<-]}(0.0,1.2)(0.0,1.2)\rput{0}(0.0,1.0){$z_{102}$}
\pscircle[linecolor=red,fillcolor=black,fillstyle=solid](0.0,1.2){0.075}
\pscircle[linecolor=red,fillcolor=black,fillstyle=solid](0.0,0.6){0.075}
\end{pspicture}
\end{center}
{\bf GE Information}:  
Carrier: [0-1:z101-.] ;  
Carrier Dual: [1-2:z101-.] ;  
Critical Boundary: 1;  
The GE above is non-degenerate.  This GE is {\em not} a leaf in the GE tree.   It has 1 valid prints (descendents).  \\[0.1in]
   It has 1 legal carrier-to-dual prints, as follows:
\begin{verbatim}
     Print 1: =0=1*<1=2*
\end{verbatim}
This completes the consideration of root-4.1.1.1, as derived from the application of a print to root-4.1.1.\\[0.1in]
\subsection*{Generalized Equation root-4.2.1.1}
\label{root-4.2.1.1}We begin from the GE root-4.2.1 (see pp. \pageref{root-4.2.1}).  {We consider its print}
\begin{verbatim}
     Print 1: =0=1*<1=2*<2=3*
\end{verbatim}
{\bf Sequence of actions in performing the Print 1:}\\
{\underline{Step 1}:} Deleted constraint between boundary 0 in (old) base [0-2:z100-.]  and boundary 1 in its dual.\\
{\underline{Step 2}:} Moved (old) base [0-1:c1-.]  to (new) boundaries 1 - 2.\\
{\underline{Step 3}:} Deleted (new) boundary 0 because it is not used inside any base.  This will cause renumbering of higher numbered boundaries.
\\[0.1in]
{Upon applying the print, the GE we obtain---which we refer to as root-4.2.1.1---is illustrated below:}
\begin{center}
\begin{pspicture}(-0.5,-0.5)(7.5,6.5)
\psline[linecolor=black]{-}(0.0,0.0)(0.0,6.0)\rput{0}(0.0,0.0){$0$}
\rput{0}(0.0,6.0){$0$}
\psline[linecolor=black]{-}(2.3333333333333335,0.0)(2.3333333333333335,6.0)\rput{0}(2.3333333333333335,0.0){$1$}
\rput{0}(2.3333333333333335,6.0){$1$}
\psline[linecolor=black]{-}(4.666666666666667,0.0)(4.666666666666667,6.0)\rput{0}(4.666666666666667,0.0){$2$}
\rput{0}(4.666666666666667,6.0){$2$}
\psline[linecolor=black]{-}(7.0,0.0)(7.0,6.0)\rput{0}(7.0,0.0){$3$}
\rput{0}(7.0,6.0){$3$}
\psline[linecolor=red]{[->}(4.666666666666667,0.75)(4.666666666666667,0.75)\rput{0}(4.666666666666667,0.55){$z_{1}$}
\psline[linecolor=red]{[->}(4.666666666666667,1.5)(4.666666666666667,1.5)\rput{0}(4.666666666666667,1.3){$z_{1}$}
\pscircle[linecolor=red,fillcolor=black,fillstyle=solid](4.666666666666667,1.5){0.075}
\pscircle[linecolor=red,fillcolor=black,fillstyle=solid](4.666666666666667,0.75){0.075}
\psline[linecolor=blue]{<-]}(0.0,0.75)(2.3333333333333335,0.75)\rput{0}(1.1666666666666667,0.55){$c_{1}$}
\psline[linecolor=blue]{<-]}(4.666666666666667,2.25)(7.0,2.25)\rput{0}(5.833333333333334,2.05){$c_{1}$}
\psline[linecolor=red]{<-]}(0.0,1.5)(2.3333333333333335,1.5)\rput{0}(1.1666666666666667,1.3){$z_{100}$}
\pscircle[linecolor=red,fillcolor=black,fillstyle=solid](0.0,1.5){0.075}
\pscircle[linecolor=red,fillcolor=black,fillstyle=solid](2.3333333333333335,3.0){0.075}
\pscircle[linecolor=red,fillcolor=white,fillstyle=solid](2.3333333333333335,1.5){0.075}
\pscircle[linecolor=red,fillcolor=white,fillstyle=solid](4.666666666666667,3.0){0.075}
\psline[linecolor=red]{<-]}(2.3333333333333335,3.0)(4.666666666666667,3.0)\rput{0}(3.5,2.8){$z_{100}$}
\psline[linecolor=red]{<-]}(2.3333333333333335,3.75)(4.666666666666667,3.75)\rput{0}(3.5,3.55){$z_{101}$}
\pscircle[linecolor=red,fillcolor=black,fillstyle=solid](2.3333333333333335,3.75){0.075}
\pscircle[linecolor=red,fillcolor=black,fillstyle=solid](4.666666666666667,4.5){0.075}
\pscircle[linecolor=red,fillcolor=white,fillstyle=solid](4.666666666666667,3.75){0.075}
\pscircle[linecolor=red,fillcolor=white,fillstyle=solid](7.0,4.5){0.075}
\psline[linecolor=red]{<-]}(4.666666666666667,4.5)(7.0,4.5)\rput{0}(5.833333333333334,4.3){$z_{101}$}
\psline[linecolor=red]{<-]}(0.0,2.25)(0.0,2.25)\rput{0}(0.0,2.05){$z_{102}$}
\pscircle[linecolor=red,fillcolor=black,fillstyle=solid](0.0,2.25){0.075}
\pscircle[linecolor=red,fillcolor=black,fillstyle=solid](0.0,3.0){0.075}
\psline[linecolor=red]{<-]}(0.0,3.0)(0.0,3.0)\rput{0}(0.0,2.8){$z_{102}$}
\end{pspicture}
\end{center}
{\bf GE Information}:  
Carrier: [0-1:z100-.] ;  
Carrier Dual: [1-2:z100-.] ;  
Critical Boundary: 1;  
The GE above is non-degenerate.  This GE is {\em not} a leaf in the GE tree.   It has 1 valid prints (descendents).  \\[0.1in]
   It has 1 legal carrier-to-dual prints, as follows:
\begin{verbatim}
     Print 1: =0=1*<1=2*
\end{verbatim}
This completes the consideration of root-4.2.1.1, as derived from the application of a print to root-4.2.1.\\[0.1in]
\newpage
\section{Cancellation scheme \#$5$}
\begin{center}
\begin{pspicture}(-0.5,-0.5)(6.5,6.5)
{\psset{fillstyle=ccslope,slopebegin=yellow!40,slopeend=gray}
\cnodeput(1.98,4.02){0}{\strut\boldmath$0$}
\cnodeput(6.00,0.00){1}{\strut\boldmath$1$}
\cnodeput(4.04,1.98){2}{\strut\boldmath$2$}
\cnodeput(0.00,6.00){3}{\strut\boldmath$3$}
}
\newcommand\arc[3]{%
  \ncline{#1}{#2}{#3}
}
\arc{-}{0}{2}{}
\arc{-}{1}{2}{}
\arc{-}{0}{3}{}
\pscurve[linecolor=red]{<<-|}(5.82,-0.18)(4.84,0.82)(3.86,1.81)(2.83,2.82)(1.81,3.84)\rput{314}(4.84,0.82){$z_{1}$}
\psline[linecolor=blue]{|->>}(4.21,2.16)(5.20,1.17)(6.18,0.18)\rput{314}(5.20,1.17){$c_{1}$}
\pscurve[linecolor=red]{|->>}(0.18,6.18)(1.17,5.19)(3.19,3.18)(4.21,2.16)\rput{315}(1.17,5.19){$z_{1}$}
\psline[linecolor=blue]{<<-|}(1.81,3.84)(0.81,4.83)(-0.18,5.82)\rput{315}(0.81,4.83){$c_{1}$}
\end{pspicture}
\end{center}
\begin{center}
\begin{tabular}{|ll|}
\hline
$z_{1}^{-1}$ & $1\leftarrow 2\leftarrow 0$\\
$c_{1}$ & $2\leftarrow 1$\\
$z_{1}$ & $3\leftarrow 0\leftarrow 2$\\
$c_{1}^{-1}$ & $0\leftarrow 3$\\
\hline
\end{tabular}
\end{center}
\subsection*{Generalized Equation root-5}
\label{root-5}Below is the root GE obtained from the cancellation diagram above.\begin{center}
\begin{pspicture}(-0.5,-0.5)(7.5,6.5)
\psline[linecolor=black]{-}(0.0,0.0)(0.0,6.0)\rput{0}(0.0,0.0){$0$}
\rput{0}(0.0,6.0){$0$}
\psline[linecolor=black]{-}(1.1666666666666667,0.0)(1.1666666666666667,6.0)\rput{0}(1.1666666666666667,0.0){$1$}
\rput{0}(1.1666666666666667,6.0){$1$}
\psline[linecolor=black]{-}(2.3333333333333335,0.0)(2.3333333333333335,6.0)\rput{0}(2.3333333333333335,0.0){$2$}
\rput{0}(2.3333333333333335,6.0){$2$}
\psline[linecolor=black]{-}(3.5,0.0)(3.5,6.0)\rput{0}(3.5,0.0){$3$}
\rput{0}(3.5,6.0){$3$}
\psline[linecolor=black]{-}(4.666666666666667,0.0)(4.666666666666667,6.0)\rput{0}(4.666666666666667,0.0){$4$}
\rput{0}(4.666666666666667,6.0){$4$}
\psline[linecolor=black]{-}(5.833333333333334,0.0)(5.833333333333334,6.0)\rput{0}(5.833333333333334,0.0){$5$}
\rput{0}(5.833333333333334,6.0){$5$}
\psline[linecolor=black]{-}(7.000000000000001,0.0)(7.000000000000001,6.0)\rput{0}(7.000000000000001,0.0){$6$}
\rput{0}(7.000000000000001,6.0){$6$}
\psline[linecolor=red]{<-]}(0.0,1.0)(2.3333333333333335,1.0)\rput{0}(1.1666666666666667,0.8){$z_{1}$}
\pscircle[linecolor=red,fillcolor=black,fillstyle=solid](0.0,1.0){0.075}
\pscircle[linecolor=red,fillcolor=black,fillstyle=solid](5.833333333333334,1.0){0.075}
\pscircle[linecolor=red,fillcolor=white,fillstyle=solid](2.3333333333333335,1.0){0.075}
\pscircle[linecolor=red,fillcolor=white,fillstyle=solid](3.5,1.0){0.075}
\psline[linecolor=red]{[->}(3.5,1.0)(5.833333333333334,1.0)\rput{0}(4.666666666666667,0.8){$z_{1}$}
\psline[linecolor=blue]{[->}(2.3333333333333335,2.0)(3.5,2.0)\rput{0}(2.916666666666667,1.8){$c_{1}$}
\psline[linecolor=blue]{<-]}(5.833333333333334,2.0)(7.0,2.0)\rput{0}(6.416666666666667,1.8){$c_{1}$}
\psline[linecolor=red]{[->}(0.0,2.0)(1.1666666666666667,2.0)\rput{0}(0.5833333333333334,1.8){$z_{100}$}
\pscircle[linecolor=red,fillcolor=black,fillstyle=solid](0.0,2.0){0.075}
\pscircle[linecolor=red,fillcolor=black,fillstyle=solid](4.666666666666667,3.0){0.075}
\pscircle[linecolor=red,fillcolor=white,fillstyle=solid](1.1666666666666667,2.0){0.075}
\pscircle[linecolor=red,fillcolor=white,fillstyle=solid](3.5,3.0){0.075}
\psline[linecolor=red]{<-]}(3.5,3.0)(4.666666666666667,3.0)\rput{0}(4.083333333333334,2.8){$z_{100}$}
\psline[linecolor=red]{[->}(1.1666666666666667,3.0)(2.3333333333333335,3.0)\rput{0}(1.75,2.8){$z_{101}$}
\pscircle[linecolor=red,fillcolor=black,fillstyle=solid](1.1666666666666667,3.0){0.075}
\pscircle[linecolor=red,fillcolor=black,fillstyle=solid](3.5,4.0){0.075}
\pscircle[linecolor=red,fillcolor=white,fillstyle=solid](2.3333333333333335,3.0){0.075}
\pscircle[linecolor=red,fillcolor=white,fillstyle=solid](2.3333333333333335,4.0){0.075}
\psline[linecolor=red]{<-]}(2.3333333333333335,4.0)(3.5,4.0)\rput{0}(2.916666666666667,3.8){$z_{101}$}
\psline[linecolor=red]{[->}(4.666666666666667,4.0)(5.833333333333334,4.0)\rput{0}(5.25,3.8){$z_{102}$}
\pscircle[linecolor=red,fillcolor=black,fillstyle=solid](4.666666666666667,4.0){0.075}
\pscircle[linecolor=red,fillcolor=black,fillstyle=solid](7.0,3.0){0.075}
\pscircle[linecolor=red,fillcolor=white,fillstyle=solid](5.833333333333334,4.0){0.075}
\pscircle[linecolor=red,fillcolor=white,fillstyle=solid](5.833333333333334,3.0){0.075}
\psline[linecolor=red]{<-]}(5.833333333333334,3.0)(7.0,3.0)\rput{0}(6.416666666666667,2.8){$z_{102}$}
\end{pspicture}
\end{center}
{\bf GE Information}:  
Carrier: [0-2:z1-.] ;  
Carrier Dual: [3-5:z1+.] ;  
Critical Boundary: 2;  
The GE above is non-degenerate.  This GE is {\em not} a leaf in the GE tree.   It has 3 valid prints (descendents).  \\[0.1in]
   It has 3 legal carrier-to-dual prints, as follows:
\begin{verbatim}
     Print 1: =0=5*<1=4*<2=3*
     Print 2: =0=5*<1<4*<2=3*
     Print 3: =0=5*<4*<1<2=3*
\end{verbatim}
We proceed.\\[0.2in]
\subsection*{Generalized Equation root-5.1}
\label{root-5.1}We begin from the GE root-5 (see pp. \pageref{root-5}).  {We consider its print}
\begin{verbatim}
     Print 1: =0=5*<1=4*<2=3*
\end{verbatim}
{\bf Sequence of actions in performing the Print 1:}\\
{\underline{Step 1}:} Moved (old) base [0-2:z1-.]  to (new) boundaries 5 - 3.\\
{\underline{Step 2}:} Moved (old) base [0-1:z100+.]  to (new) boundaries 5 - 4.\\
{\underline{Step 3}:} Moved (old) base [1-2:z101+.]  to (new) boundaries 4 - 3.\\
{\underline{Step 4}:} Collapsed (new) base [3-5:z1+.]  to the empty base (5,5).
\\
{\underline{Step 5}:} Deleted (new) boundary 0 because it is not used inside any base.  This will cause renumbering of higher numbered boundaries.
\\
{\underline{Step 6}:} Deleted (new) boundary 1 because it is not used inside any base.  This will cause renumbering of higher numbered boundaries.
\\[0.1in]
{Upon applying the print, the GE we obtain---which we refer to as root-5.1---is illustrated below:}
\begin{center}
\begin{pspicture}(-0.5,-0.5)(7.5,6.5)
\psline[linecolor=black]{-}(0.0,0.0)(0.0,6.0)\rput{0}(0.0,0.0){$0$}
\rput{0}(0.0,6.0){$0$}
\psline[linecolor=black]{-}(1.75,0.0)(1.75,6.0)\rput{0}(1.75,0.0){$1$}
\rput{0}(1.75,6.0){$1$}
\psline[linecolor=black]{-}(3.5,0.0)(3.5,6.0)\rput{0}(3.5,0.0){$2$}
\rput{0}(3.5,6.0){$2$}
\psline[linecolor=black]{-}(5.25,0.0)(5.25,6.0)\rput{0}(5.25,0.0){$3$}
\rput{0}(5.25,6.0){$3$}
\psline[linecolor=black]{-}(7.0,0.0)(7.0,6.0)\rput{0}(7.0,0.0){$4$}
\rput{0}(7.0,6.0){$4$}
\psline[linecolor=red]{[->}(5.25,0.75)(5.25,0.75)\rput{0}(5.25,0.55){$z_{1}$}
\pscircle[linecolor=red,fillcolor=black,fillstyle=solid](5.25,0.75){0.075}
\pscircle[linecolor=red,fillcolor=black,fillstyle=solid](5.25,1.5){0.075}
\psline[linecolor=red]{[->}(5.25,1.5)(5.25,1.5)\rput{0}(5.25,1.3){$z_{1}$}
\psline[linecolor=blue]{[->}(0.0,0.75)(1.75,0.75)\rput{0}(0.875,0.55){$c_{1}$}
\psline[linecolor=blue]{<-]}(5.25,2.25)(7.0,2.25)\rput{0}(6.125,2.05){$c_{1}$}
\psline[linecolor=red]{<-]}(3.5,3.0)(5.25,3.0)\rput{0}(4.375,2.8){$z_{100}$}
\psline[linecolor=red]{<-]}(1.75,1.5)(3.5,1.5)\rput{0}(2.625,1.3){$z_{100}$}
\pscircle[linecolor=red,fillcolor=black,fillstyle=solid](1.75,1.5){0.075}
\pscircle[linecolor=red,fillcolor=black,fillstyle=solid](3.5,3.0){0.075}
\pscircle[linecolor=red,fillcolor=white,fillstyle=solid](3.5,1.5){0.075}
\pscircle[linecolor=red,fillcolor=white,fillstyle=solid](5.25,3.0){0.075}
\psline[linecolor=red]{<-]}(1.75,2.25)(3.5,2.25)\rput{0}(2.625,2.05){$z_{101}$}
\psline[linecolor=red]{<-]}(0.0,3.0)(1.75,3.0)\rput{0}(0.875,2.8){$z_{101}$}
\pscircle[linecolor=red,fillcolor=black,fillstyle=solid](0.0,3.0){0.075}
\pscircle[linecolor=red,fillcolor=black,fillstyle=solid](1.75,2.25){0.075}
\pscircle[linecolor=red,fillcolor=white,fillstyle=solid](1.75,3.0){0.075}
\pscircle[linecolor=red,fillcolor=white,fillstyle=solid](3.5,2.25){0.075}
\psline[linecolor=red]{[->}(3.5,3.75)(5.25,3.75)\rput{0}(4.375,3.55){$z_{102}$}
\pscircle[linecolor=red,fillcolor=black,fillstyle=solid](3.5,3.75){0.075}
\pscircle[linecolor=red,fillcolor=black,fillstyle=solid](7.0,4.5){0.075}
\pscircle[linecolor=red,fillcolor=white,fillstyle=solid](5.25,3.75){0.075}
\pscircle[linecolor=red,fillcolor=white,fillstyle=solid](5.25,4.5){0.075}
\psline[linecolor=red]{<-]}(5.25,4.5)(7.0,4.5)\rput{0}(6.125,4.3){$z_{102}$}
\end{pspicture}
\end{center}
{\bf GE Information}:  
Carrier: [0-1:z101-.] ;  
Carrier Dual: [1-2:z101-.] ;  
Critical Boundary: 1;  
The GE above is non-degenerate.  This GE is {\em not} a leaf in the GE tree.   It has 1 valid prints (descendents).  \\[0.1in]
   It has 1 legal carrier-to-dual prints, as follows:
\begin{verbatim}
     Print 1: =0=1*<1=2*
\end{verbatim}
This completes the consideration of root-5.1, as derived from the application of a print to root-5.\\[0.1in]
\subsection*{Generalized Equation root-5.2}
\label{root-5.2}We begin from the GE root-5 (see pp. \pageref{root-5}).  {We consider its print}
\begin{verbatim}
     Print 2: =0=5*<1<4*<2=3*
\end{verbatim}
{\bf Sequence of actions in performing the Print 2:}\\
{\underline{Step 1}:} Added (new) boundary 5.\\
{\underline{Step 2}:} Moved (old) base [0-2:z1-.]  to (new) boundaries 6 - 3.\\
{\underline{Step 3}:} Moved (old) base [0-1:z100+.]  to (new) boundaries 6 - 5.\\
{\underline{Step 4}:} Moved (old) base [1-2:z101+.]  to (new) boundaries 5 - 3.\\
{\underline{Step 5}:} Collapsed (new) base [3-6:z1+.]  to the empty base (6,6).
\\
{\underline{Step 6}:} Deleted (new) boundary 0 because it is not used inside any base.  This will cause renumbering of higher numbered boundaries.
\\
{\underline{Step 7}:} Deleted (new) boundary 1 because it is not used inside any base.  This will cause renumbering of higher numbered boundaries.
\\[0.1in]
{Upon applying the print, the GE we obtain---which we refer to as root-5.2---is illustrated below:}
\begin{center}
\begin{pspicture}(-0.5,-0.5)(7.5,6.5)
\psline[linecolor=black]{-}(0.0,0.0)(0.0,6.0)\rput{0}(0.0,0.0){$0$}
\rput{0}(0.0,6.0){$0$}
\psline[linecolor=black]{-}(1.4,0.0)(1.4,6.0)\rput{0}(1.4,0.0){$1$}
\rput{0}(1.4,6.0){$1$}
\psline[linecolor=black]{-}(2.8,0.0)(2.8,6.0)\rput{0}(2.8,0.0){$2$}
\rput{0}(2.8,6.0){$2$}
\psline[linecolor=black]{-}(4.199999999999999,0.0)(4.199999999999999,6.0)\rput{0}(4.199999999999999,0.0){$3$}
\rput{0}(4.199999999999999,6.0){$3$}
\psline[linecolor=black]{-}(5.6,0.0)(5.6,6.0)\rput{0}(5.6,0.0){$4$}
\rput{0}(5.6,6.0){$4$}
\psline[linecolor=black]{-}(7.0,0.0)(7.0,6.0)\rput{0}(7.0,0.0){$5$}
\rput{0}(7.0,6.0){$5$}
\psline[linecolor=red]{[->}(5.6,0.75)(5.6,0.75)\rput{0}(5.6,0.55){$z_{1}$}
\pscircle[linecolor=red,fillcolor=black,fillstyle=solid](5.6,0.75){0.075}
\pscircle[linecolor=red,fillcolor=black,fillstyle=solid](5.6,1.5){0.075}
\psline[linecolor=red]{[->}(5.6,1.5)(5.6,1.5)\rput{0}(5.6,1.3){$z_{1}$}
\psline[linecolor=blue]{[->}(0.0,0.75)(1.4,0.75)\rput{0}(0.7,0.55){$c_{1}$}
\psline[linecolor=blue]{<-]}(5.6,2.25)(7.0,2.25)\rput{0}(6.3,2.05){$c_{1}$}
\psline[linecolor=red]{<-]}(4.199999999999999,3.0)(5.6,3.0)\rput{0}(4.8999999999999995,2.8){$z_{100}$}
\psline[linecolor=red]{<-]}(1.4,1.5)(2.8,1.5)\rput{0}(2.0999999999999996,1.3){$z_{100}$}
\pscircle[linecolor=red,fillcolor=black,fillstyle=solid](1.4,1.5){0.075}
\pscircle[linecolor=red,fillcolor=black,fillstyle=solid](4.199999999999999,3.0){0.075}
\pscircle[linecolor=red,fillcolor=white,fillstyle=solid](2.8,1.5){0.075}
\pscircle[linecolor=red,fillcolor=white,fillstyle=solid](5.6,3.0){0.075}
\psline[linecolor=red]{<-]}(1.4,2.25)(4.199999999999999,2.25)\rput{0}(2.8,2.05){$z_{101}$}
\psline[linecolor=red]{<-]}(0.0,3.0)(1.4,3.0)\rput{0}(0.7,2.8){$z_{101}$}
\pscircle[linecolor=red,fillcolor=black,fillstyle=solid](0.0,3.0){0.075}
\pscircle[linecolor=red,fillcolor=black,fillstyle=solid](1.4,2.25){0.075}
\pscircle[linecolor=red,fillcolor=white,fillstyle=solid](1.4,3.0){0.075}
\pscircle[linecolor=red,fillcolor=white,fillstyle=solid](4.199999999999999,2.25){0.075}
\psline[linecolor=red]{[->}(2.8,3.75)(5.6,3.75)\rput{0}(4.199999999999999,3.55){$z_{102}$}
\pscircle[linecolor=red,fillcolor=black,fillstyle=solid](2.8,3.75){0.075}
\pscircle[linecolor=red,fillcolor=black,fillstyle=solid](7.0,4.5){0.075}
\pscircle[linecolor=red,fillcolor=white,fillstyle=solid](5.6,3.75){0.075}
\pscircle[linecolor=red,fillcolor=white,fillstyle=solid](5.6,4.5){0.075}
\psline[linecolor=red]{<-]}(5.6,4.5)(7.0,4.5)\rput{0}(6.3,4.3){$z_{102}$}
\end{pspicture}
\end{center}
{\bf GE Information}:  
Carrier: [0-1:z101-.] ;  
Carrier Dual: [1-3:z101-.] ;  
Critical Boundary: 1;  
Observe the following facts about this GE:
The base [0-1:z101-.]  has constraints with its dual that stretch the constant segment 0 - 1 to length different from 1.  The base [4-5:z102-.]  has constraints with its dual that stretch the constant segment 4 - 5 to length different from 1.  These observations show that the GE above is degenerate.  This GE is a leaf in the GE tree.  This branch of the tree has led us to a dead end.\\[0.1in]
This completes the consideration of root-5.2, as derived from the application of a print to root-5.\\[0.1in]
\subsection*{Generalized Equation root-5.3}
\label{root-5.3}We begin from the GE root-5 (see pp. \pageref{root-5}).  {We consider its print}
\begin{verbatim}
     Print 3: =0=5*<4*<1<2=3*
\end{verbatim}
{\bf Sequence of actions in performing the Print 3:}\\
{\underline{Step 1}:} Added (new) boundary 4.\\
{\underline{Step 2}:} Moved (old) base [0-2:z1-.]  to (new) boundaries 6 - 3.\\
{\underline{Step 3}:} Moved (old) base [0-1:z100+.]  to (new) boundaries 6 - 4.\\
{\underline{Step 4}:} Moved (old) base [1-2:z101+.]  to (new) boundaries 4 - 3.\\
{\underline{Step 5}:} Collapsed (new) base [3-6:z1+.]  to the empty base (6,6).
\\
{\underline{Step 6}:} Deleted (new) boundary 0 because it is not used inside any base.  This will cause renumbering of higher numbered boundaries.
\\
{\underline{Step 7}:} Deleted (new) boundary 1 because it is not used inside any base.  This will cause renumbering of higher numbered boundaries.
\\[0.1in]
{Upon applying the print, the GE we obtain---which we refer to as root-5.3---is illustrated below:}
\begin{center}
\begin{pspicture}(-0.5,-0.5)(7.5,6.5)
\psline[linecolor=black]{-}(0.0,0.0)(0.0,6.0)\rput{0}(0.0,0.0){$0$}
\rput{0}(0.0,6.0){$0$}
\psline[linecolor=black]{-}(1.4,0.0)(1.4,6.0)\rput{0}(1.4,0.0){$1$}
\rput{0}(1.4,6.0){$1$}
\psline[linecolor=black]{-}(2.8,0.0)(2.8,6.0)\rput{0}(2.8,0.0){$2$}
\rput{0}(2.8,6.0){$2$}
\psline[linecolor=black]{-}(4.199999999999999,0.0)(4.199999999999999,6.0)\rput{0}(4.199999999999999,0.0){$3$}
\rput{0}(4.199999999999999,6.0){$3$}
\psline[linecolor=black]{-}(5.6,0.0)(5.6,6.0)\rput{0}(5.6,0.0){$4$}
\rput{0}(5.6,6.0){$4$}
\psline[linecolor=black]{-}(7.0,0.0)(7.0,6.0)\rput{0}(7.0,0.0){$5$}
\rput{0}(7.0,6.0){$5$}
\psline[linecolor=red]{[->}(5.6,0.75)(5.6,0.75)\rput{0}(5.6,0.55){$z_{1}$}
\psline[linecolor=red]{[->}(5.6,1.5)(5.6,1.5)\rput{0}(5.6,1.3){$z_{1}$}
\pscircle[linecolor=red,fillcolor=black,fillstyle=solid](5.6,1.5){0.075}
\pscircle[linecolor=red,fillcolor=black,fillstyle=solid](5.6,0.75){0.075}
\psline[linecolor=blue]{[->}(0.0,0.75)(1.4,0.75)\rput{0}(0.7,0.55){$c_{1}$}
\psline[linecolor=blue]{<-]}(5.6,2.25)(7.0,2.25)\rput{0}(6.3,2.05){$c_{1}$}
\psline[linecolor=red]{<-]}(2.8,3.0)(5.6,3.0)\rput{0}(4.199999999999999,2.8){$z_{100}$}
\psline[linecolor=red]{<-]}(1.4,1.5)(4.199999999999999,1.5)\rput{0}(2.8,1.3){$z_{100}$}
\pscircle[linecolor=red,fillcolor=black,fillstyle=solid](1.4,1.5){0.075}
\pscircle[linecolor=red,fillcolor=black,fillstyle=solid](2.8,3.0){0.075}
\pscircle[linecolor=red,fillcolor=white,fillstyle=solid](4.199999999999999,1.5){0.075}
\pscircle[linecolor=red,fillcolor=white,fillstyle=solid](5.6,3.0){0.075}
\psline[linecolor=red]{<-]}(1.4,2.25)(2.8,2.25)\rput{0}(2.0999999999999996,2.05){$z_{101}$}
\psline[linecolor=red]{<-]}(0.0,3.0)(1.4,3.0)\rput{0}(0.7,2.8){$z_{101}$}
\pscircle[linecolor=red,fillcolor=black,fillstyle=solid](0.0,3.0){0.075}
\pscircle[linecolor=red,fillcolor=black,fillstyle=solid](1.4,2.25){0.075}
\pscircle[linecolor=red,fillcolor=white,fillstyle=solid](1.4,3.0){0.075}
\pscircle[linecolor=red,fillcolor=white,fillstyle=solid](2.8,2.25){0.075}
\psline[linecolor=red]{[->}(4.199999999999999,3.75)(5.6,3.75)\rput{0}(4.8999999999999995,3.55){$z_{102}$}
\pscircle[linecolor=red,fillcolor=black,fillstyle=solid](4.199999999999999,3.75){0.075}
\pscircle[linecolor=red,fillcolor=black,fillstyle=solid](7.0,4.5){0.075}
\pscircle[linecolor=red,fillcolor=white,fillstyle=solid](5.6,3.75){0.075}
\pscircle[linecolor=red,fillcolor=white,fillstyle=solid](5.6,4.5){0.075}
\psline[linecolor=red]{<-]}(5.6,4.5)(7.0,4.5)\rput{0}(6.3,4.3){$z_{102}$}
\end{pspicture}
\end{center}
{\bf GE Information}:  
Carrier: [0-1:z101-.] ;  
Carrier Dual: [1-2:z101-.] ;  
Critical Boundary: 1;  
The GE above is non-degenerate.  This GE is {\em not} a leaf in the GE tree.   It has 1 valid prints (descendents).  \\[0.1in]
   It has 1 legal carrier-to-dual prints, as follows:
\begin{verbatim}
     Print 1: =0=1*<1=2*
\end{verbatim}
This completes the consideration of root-5.3, as derived from the application of a print to root-5.\\[0.1in]
\subsection*{Generalized Equation root-5.1.1}
\label{root-5.1.1}We begin from the GE root-5.1 (see pp. \pageref{root-5.1}).  {We consider its print}
\begin{verbatim}
     Print 1: =0=1*<1=2*
\end{verbatim}
{\bf Sequence of actions in performing the Print 1:}\\
{\underline{Step 1}:} Moved (old) base [0-1:z101-.]  to (new) boundaries 1 - 2.\\
{\underline{Step 2}:} Moved (old) base [0-1:c1+.]  to (new) boundaries 1 - 2.\\
{\underline{Step 3}:} Collapsed (new) base [1-2:z101-.]  to the empty base (2,2).
\\
{\underline{Step 4}:} Deleted (new) boundary 0 because it is not used inside any base.  This will cause renumbering of higher numbered boundaries.
\\[0.1in]
{Upon applying the print, the GE we obtain---which we refer to as root-5.1.1---is illustrated below:}
\begin{center}
\begin{pspicture}(-0.5,-0.5)(7.5,6.5)
\psline[linecolor=black]{-}(0.0,0.0)(0.0,6.0)\rput{0}(0.0,0.0){$0$}
\rput{0}(0.0,6.0){$0$}
\psline[linecolor=black]{-}(2.3333333333333335,0.0)(2.3333333333333335,6.0)\rput{0}(2.3333333333333335,0.0){$1$}
\rput{0}(2.3333333333333335,6.0){$1$}
\psline[linecolor=black]{-}(4.666666666666667,0.0)(4.666666666666667,6.0)\rput{0}(4.666666666666667,0.0){$2$}
\rput{0}(4.666666666666667,6.0){$2$}
\psline[linecolor=black]{-}(7.0,0.0)(7.0,6.0)\rput{0}(7.0,0.0){$3$}
\rput{0}(7.0,6.0){$3$}
\psline[linecolor=red]{[->}(4.666666666666667,0.75)(4.666666666666667,0.75)\rput{0}(4.666666666666667,0.55){$z_{1}$}
\pscircle[linecolor=red,fillcolor=black,fillstyle=solid](4.666666666666667,0.75){0.075}
\pscircle[linecolor=red,fillcolor=black,fillstyle=solid](4.666666666666667,1.5){0.075}
\psline[linecolor=red]{[->}(4.666666666666667,1.5)(4.666666666666667,1.5)\rput{0}(4.666666666666667,1.3){$z_{1}$}
\psline[linecolor=blue]{[->}(0.0,0.75)(2.3333333333333335,0.75)\rput{0}(1.1666666666666667,0.55){$c_{1}$}
\psline[linecolor=blue]{<-]}(4.666666666666667,2.25)(7.0,2.25)\rput{0}(5.833333333333334,2.05){$c_{1}$}
\psline[linecolor=red]{<-]}(2.3333333333333335,3.0)(4.666666666666667,3.0)\rput{0}(3.5,2.8){$z_{100}$}
\psline[linecolor=red]{<-]}(0.0,1.5)(2.3333333333333335,1.5)\rput{0}(1.1666666666666667,1.3){$z_{100}$}
\pscircle[linecolor=red,fillcolor=black,fillstyle=solid](0.0,1.5){0.075}
\pscircle[linecolor=red,fillcolor=black,fillstyle=solid](2.3333333333333335,3.0){0.075}
\pscircle[linecolor=red,fillcolor=white,fillstyle=solid](2.3333333333333335,1.5){0.075}
\pscircle[linecolor=red,fillcolor=white,fillstyle=solid](4.666666666666667,3.0){0.075}
\psline[linecolor=red]{<-]}(2.3333333333333335,2.25)(2.3333333333333335,2.25)\rput{0}(2.3333333333333335,2.05){$z_{101}$}
\pscircle[linecolor=red,fillcolor=black,fillstyle=solid](2.3333333333333335,2.25){0.075}
\pscircle[linecolor=red,fillcolor=black,fillstyle=solid](2.3333333333333335,3.75){0.075}
\psline[linecolor=red]{<-]}(2.3333333333333335,3.75)(2.3333333333333335,3.75)\rput{0}(2.3333333333333335,3.55){$z_{101}$}
\psline[linecolor=red]{[->}(2.3333333333333335,4.5)(4.666666666666667,4.5)\rput{0}(3.5,4.3){$z_{102}$}
\pscircle[linecolor=red,fillcolor=black,fillstyle=solid](2.3333333333333335,4.5){0.075}
\pscircle[linecolor=red,fillcolor=black,fillstyle=solid](7.0,3.75){0.075}
\pscircle[linecolor=red,fillcolor=white,fillstyle=solid](4.666666666666667,4.5){0.075}
\pscircle[linecolor=red,fillcolor=white,fillstyle=solid](4.666666666666667,3.75){0.075}
\psline[linecolor=red]{<-]}(4.666666666666667,3.75)(7.0,3.75)\rput{0}(5.833333333333334,3.55){$z_{102}$}
\end{pspicture}
\end{center}
{\bf GE Information}:  
Carrier: [0-1:z100-.] ;  
Carrier Dual: [1-2:z100-.] ;  
Critical Boundary: 1;  
The GE above is non-degenerate.  This GE is {\em not} a leaf in the GE tree.   It has 1 valid prints (descendents).  \\[0.1in]
   It has 1 legal carrier-to-dual prints, as follows:
\begin{verbatim}
     Print 1: =0=1*<1=2*
\end{verbatim}
This completes the consideration of root-5.1.1, as derived from the application of a print to root-5.1.\\[0.1in]
\subsection*{Generalized Equation root-5.3.1}
\label{root-5.3.1}We begin from the GE root-5.3 (see pp. \pageref{root-5.3}).  {We consider its print}
\begin{verbatim}
     Print 1: =0=1*<1=2*
\end{verbatim}
{\bf Sequence of actions in performing the Print 1:}\\
{\underline{Step 1}:} Moved (old) base [0-1:z101-.]  to (new) boundaries 1 - 2.\\
{\underline{Step 2}:} Moved (old) base [0-1:c1+.]  to (new) boundaries 1 - 2.\\
{\underline{Step 3}:} Collapsed (new) base [1-2:z101-.]  to the empty base (2,2).
\\
{\underline{Step 4}:} Deleted (new) boundary 0 because it is not used inside any base.  This will cause renumbering of higher numbered boundaries.
\\[0.1in]
{Upon applying the print, the GE we obtain---which we refer to as root-5.3.1---is illustrated below:}
\begin{center}
\begin{pspicture}(-0.5,-0.5)(7.5,6.5)
\psline[linecolor=black]{-}(0.0,0.0)(0.0,6.0)\rput{0}(0.0,0.0){$0$}
\rput{0}(0.0,6.0){$0$}
\psline[linecolor=black]{-}(1.75,0.0)(1.75,6.0)\rput{0}(1.75,0.0){$1$}
\rput{0}(1.75,6.0){$1$}
\psline[linecolor=black]{-}(3.5,0.0)(3.5,6.0)\rput{0}(3.5,0.0){$2$}
\rput{0}(3.5,6.0){$2$}
\psline[linecolor=black]{-}(5.25,0.0)(5.25,6.0)\rput{0}(5.25,0.0){$3$}
\rput{0}(5.25,6.0){$3$}
\psline[linecolor=black]{-}(7.0,0.0)(7.0,6.0)\rput{0}(7.0,0.0){$4$}
\rput{0}(7.0,6.0){$4$}
\psline[linecolor=red]{[->}(5.25,0.75)(5.25,0.75)\rput{0}(5.25,0.55){$z_{1}$}
\psline[linecolor=red]{[->}(5.25,1.5)(5.25,1.5)\rput{0}(5.25,1.3){$z_{1}$}
\pscircle[linecolor=red,fillcolor=black,fillstyle=solid](5.25,1.5){0.075}
\pscircle[linecolor=red,fillcolor=black,fillstyle=solid](5.25,0.75){0.075}
\psline[linecolor=blue]{[->}(0.0,0.75)(1.75,0.75)\rput{0}(0.875,0.55){$c_{1}$}
\psline[linecolor=blue]{<-]}(5.25,2.25)(7.0,2.25)\rput{0}(6.125,2.05){$c_{1}$}
\psline[linecolor=red]{<-]}(1.75,3.0)(5.25,3.0)\rput{0}(3.5,2.8){$z_{100}$}
\psline[linecolor=red]{<-]}(0.0,1.5)(3.5,1.5)\rput{0}(1.75,1.3){$z_{100}$}
\pscircle[linecolor=red,fillcolor=black,fillstyle=solid](0.0,1.5){0.075}
\pscircle[linecolor=red,fillcolor=black,fillstyle=solid](1.75,3.0){0.075}
\pscircle[linecolor=red,fillcolor=white,fillstyle=solid](3.5,1.5){0.075}
\pscircle[linecolor=red,fillcolor=white,fillstyle=solid](5.25,3.0){0.075}
\psline[linecolor=red]{<-]}(1.75,2.25)(1.75,2.25)\rput{0}(1.75,2.05){$z_{101}$}
\psline[linecolor=red]{<-]}(1.75,3.75)(1.75,3.75)\rput{0}(1.75,3.55){$z_{101}$}
\pscircle[linecolor=red,fillcolor=black,fillstyle=solid](1.75,3.75){0.075}
\pscircle[linecolor=red,fillcolor=black,fillstyle=solid](1.75,2.25){0.075}
\psline[linecolor=red]{[->}(3.5,3.75)(5.25,3.75)\rput{0}(4.375,3.55){$z_{102}$}
\pscircle[linecolor=red,fillcolor=black,fillstyle=solid](3.5,3.75){0.075}
\pscircle[linecolor=red,fillcolor=black,fillstyle=solid](7.0,4.5){0.075}
\pscircle[linecolor=red,fillcolor=white,fillstyle=solid](5.25,3.75){0.075}
\pscircle[linecolor=red,fillcolor=white,fillstyle=solid](5.25,4.5){0.075}
\psline[linecolor=red]{<-]}(5.25,4.5)(7.0,4.5)\rput{0}(6.125,4.3){$z_{102}$}
\end{pspicture}
\end{center}
{\bf GE Information}:  
Carrier: [0-2:z100-.] ;  
Carrier Dual: [1-3:z100-.] ;  
Critical Boundary: 1;  
The GE above is non-degenerate.  This GE is {\em not} a leaf in the GE tree.   It has 2 valid prints (descendents).  \\[0.1in]
   It has 2 legal carrier-to-dual prints, as follows:
\begin{verbatim}
     Print 1: =0=1*<1=2*<2=3*
     Print 2: =0=1*<1<2*<2=3*
\end{verbatim}
This completes the consideration of root-5.3.1, as derived from the application of a print to root-5.3.\\[0.1in]
\subsection*{Generalized Equation root-5.1.1.1}
\label{root-5.1.1.1}We begin from the GE root-5.1.1 (see pp. \pageref{root-5.1.1}).  {We consider its print}
\begin{verbatim}
     Print 1: =0=1*<1=2*
\end{verbatim}
{\bf Sequence of actions in performing the Print 1:}\\
{\underline{Step 1}:} Moved (old) base [0-1:z100-.]  to (new) boundaries 1 - 2.\\
{\underline{Step 2}:} Moved (old) base [0-1:c1+.]  to (new) boundaries 1 - 2.\\
{\underline{Step 3}:} Collapsed (new) base [1-2:z100-.]  to the empty base (2,2).
\\
{\underline{Step 4}:} Deleted (new) boundary 0 because it is not used inside any base.  This will cause renumbering of higher numbered boundaries.
\\[0.1in]
{Upon applying the print, the GE we obtain---which we refer to as root-5.1.1.1---is illustrated below:}
\begin{center}
\begin{pspicture}(-0.5,-0.5)(7.5,6.5)
\psline[linecolor=black]{-}(0.0,0.0)(0.0,6.0)\rput{0}(0.0,0.0){$0$}
\rput{0}(0.0,6.0){$0$}
\psline[linecolor=black]{-}(3.5,0.0)(3.5,6.0)\rput{0}(3.5,0.0){$1$}
\rput{0}(3.5,6.0){$1$}
\psline[linecolor=black]{-}(7.0,0.0)(7.0,6.0)\rput{0}(7.0,0.0){$2$}
\rput{0}(7.0,6.0){$2$}
\psline[linecolor=red]{[->}(3.5,0.6)(3.5,0.6)\rput{0}(3.5,0.39999999999999997){$z_{1}$}
\pscircle[linecolor=red,fillcolor=black,fillstyle=solid](3.5,0.6){0.075}
\pscircle[linecolor=red,fillcolor=black,fillstyle=solid](3.5,1.2){0.075}
\psline[linecolor=red]{[->}(3.5,1.2)(3.5,1.2)\rput{0}(3.5,1.0){$z_{1}$}
\psline[linecolor=blue]{[->}(0.0,1.7999999999999998)(3.5,1.7999999999999998)\rput{0}(1.75,1.5999999999999999){$c_{1}$}
\psline[linecolor=blue]{<-]}(3.5,2.4)(7.0,2.4)\rput{0}(5.25,2.1999999999999997){$c_{1}$}
\psline[linecolor=red]{<-]}(3.5,3.0)(3.5,3.0)\rput{0}(3.5,2.8){$z_{100}$}
\psline[linecolor=red]{<-]}(3.5,3.5999999999999996)(3.5,3.5999999999999996)\rput{0}(3.5,3.3999999999999995){$z_{100}$}
\pscircle[linecolor=red,fillcolor=black,fillstyle=solid](3.5,3.5999999999999996){0.075}
\pscircle[linecolor=red,fillcolor=black,fillstyle=solid](3.5,3.0){0.075}
\psline[linecolor=red]{<-]}(0.0,0.6)(0.0,0.6)\rput{0}(0.0,0.39999999999999997){$z_{101}$}
\pscircle[linecolor=red,fillcolor=black,fillstyle=solid](0.0,0.6){0.075}
\pscircle[linecolor=red,fillcolor=black,fillstyle=solid](0.0,1.2){0.075}
\psline[linecolor=red]{<-]}(0.0,1.2)(0.0,1.2)\rput{0}(0.0,1.0){$z_{101}$}
\psline[linecolor=red]{[->}(0.0,4.2)(3.5,4.2)\rput{0}(1.75,4.0){$z_{102}$}
\pscircle[linecolor=red,fillcolor=black,fillstyle=solid](0.0,4.2){0.075}
\pscircle[linecolor=red,fillcolor=black,fillstyle=solid](7.0,4.8){0.075}
\pscircle[linecolor=red,fillcolor=white,fillstyle=solid](3.5,4.2){0.075}
\pscircle[linecolor=red,fillcolor=white,fillstyle=solid](3.5,4.8){0.075}
\psline[linecolor=red]{<-]}(3.5,4.8)(7.0,4.8)\rput{0}(5.25,4.6){$z_{102}$}
\end{pspicture}
\end{center}
{\bf GE Information}:  
Carrier: [0-1:z102+.] ;  
Carrier Dual: [1-2:z102-.] ;  
Critical Boundary: 1;  
The GE above is non-degenerate.  This GE is {\em not} a leaf in the GE tree.   It has 1 valid prints (descendents).  \\[0.1in]
   It has 1 legal carrier-to-dual prints, as follows:
\begin{verbatim}
     Print 1: =0=2*<1=1*
\end{verbatim}
This completes the consideration of root-5.1.1.1, as derived from the application of a print to root-5.1.1.\\[0.1in]
\subsection*{Generalized Equation root-5.3.1.1}
\label{root-5.3.1.1}We begin from the GE root-5.3.1 (see pp. \pageref{root-5.3.1}).  {We consider its print}
\begin{verbatim}
     Print 1: =0=1*<1=2*<2=3*
\end{verbatim}
{\bf Sequence of actions in performing the Print 1:}\\
{\underline{Step 1}:} Deleted constraint between boundary 0 in (old) base [0-2:z100-.]  and boundary 1 in its dual.\\
{\underline{Step 2}:} Moved (old) base [0-1:c1+.]  to (new) boundaries 1 - 2.\\
{\underline{Step 3}:} Deleted (new) boundary 0 because it is not used inside any base.  This will cause renumbering of higher numbered boundaries.
\\[0.1in]
{Upon applying the print, the GE we obtain---which we refer to as root-5.3.1.1---is illustrated below:}
\begin{center}
\begin{pspicture}(-0.5,-0.5)(7.5,6.5)
\psline[linecolor=black]{-}(0.0,0.0)(0.0,6.0)\rput{0}(0.0,0.0){$0$}
\rput{0}(0.0,6.0){$0$}
\psline[linecolor=black]{-}(2.3333333333333335,0.0)(2.3333333333333335,6.0)\rput{0}(2.3333333333333335,0.0){$1$}
\rput{0}(2.3333333333333335,6.0){$1$}
\psline[linecolor=black]{-}(4.666666666666667,0.0)(4.666666666666667,6.0)\rput{0}(4.666666666666667,0.0){$2$}
\rput{0}(4.666666666666667,6.0){$2$}
\psline[linecolor=black]{-}(7.0,0.0)(7.0,6.0)\rput{0}(7.0,0.0){$3$}
\rput{0}(7.0,6.0){$3$}
\psline[linecolor=red]{[->}(4.666666666666667,0.75)(4.666666666666667,0.75)\rput{0}(4.666666666666667,0.55){$z_{1}$}
\psline[linecolor=red]{[->}(4.666666666666667,1.5)(4.666666666666667,1.5)\rput{0}(4.666666666666667,1.3){$z_{1}$}
\pscircle[linecolor=red,fillcolor=black,fillstyle=solid](4.666666666666667,1.5){0.075}
\pscircle[linecolor=red,fillcolor=black,fillstyle=solid](4.666666666666667,0.75){0.075}
\psline[linecolor=blue]{[->}(0.0,0.75)(2.3333333333333335,0.75)\rput{0}(1.1666666666666667,0.55){$c_{1}$}
\psline[linecolor=blue]{<-]}(4.666666666666667,2.25)(7.0,2.25)\rput{0}(5.833333333333334,2.05){$c_{1}$}
\psline[linecolor=red]{<-]}(2.3333333333333335,3.0)(4.666666666666667,3.0)\rput{0}(3.5,2.8){$z_{100}$}
\psline[linecolor=red]{<-]}(0.0,1.5)(2.3333333333333335,1.5)\rput{0}(1.1666666666666667,1.3){$z_{100}$}
\pscircle[linecolor=red,fillcolor=black,fillstyle=solid](0.0,1.5){0.075}
\pscircle[linecolor=red,fillcolor=black,fillstyle=solid](2.3333333333333335,3.0){0.075}
\pscircle[linecolor=red,fillcolor=white,fillstyle=solid](2.3333333333333335,1.5){0.075}
\pscircle[linecolor=red,fillcolor=white,fillstyle=solid](4.666666666666667,3.0){0.075}
\psline[linecolor=red]{<-]}(0.0,2.25)(0.0,2.25)\rput{0}(0.0,2.05){$z_{101}$}
\psline[linecolor=red]{<-]}(0.0,3.0)(0.0,3.0)\rput{0}(0.0,2.8){$z_{101}$}
\pscircle[linecolor=red,fillcolor=black,fillstyle=solid](0.0,3.0){0.075}
\pscircle[linecolor=red,fillcolor=black,fillstyle=solid](0.0,2.25){0.075}
\psline[linecolor=red]{[->}(2.3333333333333335,3.75)(4.666666666666667,3.75)\rput{0}(3.5,3.55){$z_{102}$}
\pscircle[linecolor=red,fillcolor=black,fillstyle=solid](2.3333333333333335,3.75){0.075}
\pscircle[linecolor=red,fillcolor=black,fillstyle=solid](7.0,4.5){0.075}
\pscircle[linecolor=red,fillcolor=white,fillstyle=solid](4.666666666666667,3.75){0.075}
\pscircle[linecolor=red,fillcolor=white,fillstyle=solid](4.666666666666667,4.5){0.075}
\psline[linecolor=red]{<-]}(4.666666666666667,4.5)(7.0,4.5)\rput{0}(5.833333333333334,4.3){$z_{102}$}
\end{pspicture}
\end{center}
{\bf GE Information}:  
Carrier: [0-1:z100-.] ;  
Carrier Dual: [1-2:z100-.] ;  
Critical Boundary: 1;  
The GE above is non-degenerate.  This GE is {\em not} a leaf in the GE tree.   It has 1 valid prints (descendents).  \\[0.1in]
   It has 1 legal carrier-to-dual prints, as follows:
\begin{verbatim}
     Print 1: =0=1*<1=2*
\end{verbatim}
This completes the consideration of root-5.3.1.1, as derived from the application of a print to root-5.3.1.\\[0.1in]
\newpage
\section{Cancellation scheme \#$6$}
\begin{center}
\begin{pspicture}(-0.5,-0.5)(6.5,6.5)
{\psset{fillstyle=ccslope,slopebegin=yellow!40,slopeend=gray}
\cnodeput(6.00,6.00){0}{\strut\boldmath$0$}
\cnodeput(0.00,0.00){1}{\strut\boldmath$1$}
\cnodeput(1.98,1.98){2}{\strut\boldmath$2$}
\cnodeput(4.02,4.07){3}{\strut\boldmath$3$}
}
\newcommand\arc[3]{%
  \ncline{#1}{#2}{#3}
}
\arc{-}{1}{2}{}
\arc{-}{2}{3}{}
\arc{-}{0}{3}{}
\pscurve[linecolor=red]{<<-|}(-0.18,0.18)(0.81,1.17)(1.80,2.16)(2.82,3.20)(3.84,4.24)(4.84,5.21)(5.83,6.18)\rput{45}(2.82,3.20){$z_{1}$}
\psline[linecolor=blue]{|->>}(2.15,1.81)(1.17,0.82)(0.18,-0.18)\rput{45}(1.17,0.82){$c_{1}$}
\psline[linecolor=red]{|->>}(4.20,3.89)(3.18,2.85)(2.16,1.81)\rput{45}(3.18,2.85){$z_{1}$}
\psline[linecolor=blue]{<<-|}(6.17,5.82)(5.19,4.85)(4.20,3.89)\rput{44}(5.19,4.85){$c_{1}$}
\end{pspicture}
\end{center}
\begin{center}
\begin{tabular}{|ll|}
\hline
$z_{1}^{-1}$ & $1\leftarrow 2\leftarrow 3\leftarrow 0$\\
$c_{1}$ & $2\leftarrow 1$\\
$z_{1}$ & $3\leftarrow 2$\\
$c_{1}^{-1}$ & $0\leftarrow 3$\\
\hline
\end{tabular}
\end{center}
\subsection*{Generalized Equation root-6}
\label{root-6}Below is the root GE obtained from the cancellation diagram above.\begin{center}
\begin{pspicture}(-0.5,-0.5)(7.5,6.5)
\psline[linecolor=black]{-}(0.0,0.0)(0.0,6.0)\rput{0}(0.0,0.0){$0$}
\rput{0}(0.0,6.0){$0$}
\psline[linecolor=black]{-}(1.1666666666666667,0.0)(1.1666666666666667,6.0)\rput{0}(1.1666666666666667,0.0){$1$}
\rput{0}(1.1666666666666667,6.0){$1$}
\psline[linecolor=black]{-}(2.3333333333333335,0.0)(2.3333333333333335,6.0)\rput{0}(2.3333333333333335,0.0){$2$}
\rput{0}(2.3333333333333335,6.0){$2$}
\psline[linecolor=black]{-}(3.5,0.0)(3.5,6.0)\rput{0}(3.5,0.0){$3$}
\rput{0}(3.5,6.0){$3$}
\psline[linecolor=black]{-}(4.666666666666667,0.0)(4.666666666666667,6.0)\rput{0}(4.666666666666667,0.0){$4$}
\rput{0}(4.666666666666667,6.0){$4$}
\psline[linecolor=black]{-}(5.833333333333334,0.0)(5.833333333333334,6.0)\rput{0}(5.833333333333334,0.0){$5$}
\rput{0}(5.833333333333334,6.0){$5$}
\psline[linecolor=black]{-}(7.000000000000001,0.0)(7.000000000000001,6.0)\rput{0}(7.000000000000001,0.0){$6$}
\rput{0}(7.000000000000001,6.0){$6$}
\psline[linecolor=red]{<-]}(0.0,1.0)(3.5,1.0)\rput{0}(1.75,0.8){$z_{1}$}
\pscircle[linecolor=red,fillcolor=black,fillstyle=solid](0.0,1.0){0.075}
\pscircle[linecolor=red,fillcolor=black,fillstyle=solid](5.833333333333334,1.0){0.075}
\pscircle[linecolor=red,fillcolor=white,fillstyle=solid](3.5,1.0){0.075}
\pscircle[linecolor=red,fillcolor=white,fillstyle=solid](4.666666666666667,1.0){0.075}
\psline[linecolor=red]{[->}(4.666666666666667,1.0)(5.833333333333334,1.0)\rput{0}(5.25,0.8){$z_{1}$}
\psline[linecolor=blue]{[->}(3.5,2.0)(4.666666666666667,2.0)\rput{0}(4.083333333333334,1.8){$c_{1}$}
\psline[linecolor=blue]{<-]}(5.833333333333334,2.0)(7.0,2.0)\rput{0}(6.416666666666667,1.8){$c_{1}$}
\psline[linecolor=red]{[->}(2.3333333333333335,3.0)(3.5,3.0)\rput{0}(2.916666666666667,2.8){$z_{100}$}
\pscircle[linecolor=red,fillcolor=black,fillstyle=solid](2.3333333333333335,3.0){0.075}
\pscircle[linecolor=red,fillcolor=black,fillstyle=solid](4.666666666666667,4.0){0.075}
\pscircle[linecolor=red,fillcolor=white,fillstyle=solid](3.5,3.0){0.075}
\pscircle[linecolor=red,fillcolor=white,fillstyle=solid](3.5,4.0){0.075}
\psline[linecolor=red]{<-]}(3.5,4.0)(4.666666666666667,4.0)\rput{0}(4.083333333333334,3.8){$z_{100}$}
\psline[linecolor=red]{[->}(1.1666666666666667,2.0)(2.3333333333333335,2.0)\rput{0}(1.75,1.8){$z_{101}$}
\pscircle[linecolor=red,fillcolor=black,fillstyle=solid](1.1666666666666667,2.0){0.075}
\pscircle[linecolor=red,fillcolor=black,fillstyle=solid](5.833333333333334,3.0){0.075}
\pscircle[linecolor=red,fillcolor=white,fillstyle=solid](2.3333333333333335,2.0){0.075}
\pscircle[linecolor=red,fillcolor=white,fillstyle=solid](4.666666666666667,3.0){0.075}
\psline[linecolor=red]{<-]}(4.666666666666667,3.0)(5.833333333333334,3.0)\rput{0}(5.25,2.8){$z_{101}$}
\psline[linecolor=red]{[->}(0.0,3.0)(1.1666666666666667,3.0)\rput{0}(0.5833333333333334,2.8){$z_{102}$}
\pscircle[linecolor=red,fillcolor=black,fillstyle=solid](0.0,3.0){0.075}
\pscircle[linecolor=red,fillcolor=black,fillstyle=solid](7.0,4.0){0.075}
\pscircle[linecolor=red,fillcolor=white,fillstyle=solid](1.1666666666666667,3.0){0.075}
\pscircle[linecolor=red,fillcolor=white,fillstyle=solid](5.833333333333334,4.0){0.075}
\psline[linecolor=red]{<-]}(5.833333333333334,4.0)(7.0,4.0)\rput{0}(6.416666666666667,3.8){$z_{102}$}
\end{pspicture}
\end{center}
{\bf GE Information}:  
Carrier: [0-3:z1-.] ;  
Carrier Dual: [4-5:z1+.] ;  
Critical Boundary: 3;  
The GE above is non-degenerate.  This GE is {\em not} a leaf in the GE tree.   It has 1 valid prints (descendents).  \\[0.1in]
   It has 1 legal carrier-to-dual prints, as follows:
\begin{verbatim}
     Print 1: =0=5*<1<2<3=4*
\end{verbatim}
We proceed.\\[0.2in]
\subsection*{Generalized Equation root-6.1}
\label{root-6.1}We begin from the GE root-6 (see pp. \pageref{root-6}).  {We consider its print}
\begin{verbatim}
     Print 1: =0=5*<1<2<3=4*
\end{verbatim}
{\bf Sequence of actions in performing the Print 1:}\\
{\underline{Step 1}:} Added (new) boundary 5.\\
{\underline{Step 2}:} Added (new) boundary 6.\\
{\underline{Step 3}:} Moved (old) base [0-3:z1-.]  to (new) boundaries 7 - 4.\\
{\underline{Step 4}:} Moved (old) base [2-3:z100+.]  to (new) boundaries 5 - 4.\\
{\underline{Step 5}:} Moved (old) base [1-2:z101+.]  to (new) boundaries 6 - 5.\\
{\underline{Step 6}:} Moved (old) base [0-1:z102+.]  to (new) boundaries 7 - 6.\\
{\underline{Step 7}:} Collapsed (new) base [4-7:z1+.]  to the empty base (7,7).
\\
{\underline{Step 8}:} Deleted (new) boundary 0 because it is not used inside any base.  This will cause renumbering of higher numbered boundaries.
\\
{\underline{Step 9}:} Deleted (new) boundary 1 because it is not used inside any base.  This will cause renumbering of higher numbered boundaries.
\\
{\underline{Step 10}:} Deleted (new) boundary 2 because it is not used inside any base.  This will cause renumbering of higher numbered boundaries.
\\[0.1in]
{Upon applying the print, the GE we obtain---which we refer to as root-6.1---is illustrated below:}
\begin{center}
\begin{pspicture}(-0.5,-0.5)(7.5,6.5)
\psline[linecolor=black]{-}(0.0,0.0)(0.0,6.0)\rput{0}(0.0,0.0){$0$}
\rput{0}(0.0,6.0){$0$}
\psline[linecolor=black]{-}(1.4,0.0)(1.4,6.0)\rput{0}(1.4,0.0){$1$}
\rput{0}(1.4,6.0){$1$}
\psline[linecolor=black]{-}(2.8,0.0)(2.8,6.0)\rput{0}(2.8,0.0){$2$}
\rput{0}(2.8,6.0){$2$}
\psline[linecolor=black]{-}(4.199999999999999,0.0)(4.199999999999999,6.0)\rput{0}(4.199999999999999,0.0){$3$}
\rput{0}(4.199999999999999,6.0){$3$}
\psline[linecolor=black]{-}(5.6,0.0)(5.6,6.0)\rput{0}(5.6,0.0){$4$}
\rput{0}(5.6,6.0){$4$}
\psline[linecolor=black]{-}(7.0,0.0)(7.0,6.0)\rput{0}(7.0,0.0){$5$}
\rput{0}(7.0,6.0){$5$}
\psline[linecolor=red]{[->}(5.6,0.75)(5.6,0.75)\rput{0}(5.6,0.55){$z_{1}$}
\pscircle[linecolor=red,fillcolor=black,fillstyle=solid](5.6,0.75){0.075}
\pscircle[linecolor=red,fillcolor=black,fillstyle=solid](5.6,1.5){0.075}
\psline[linecolor=red]{[->}(5.6,1.5)(5.6,1.5)\rput{0}(5.6,1.3){$z_{1}$}
\psline[linecolor=blue]{[->}(0.0,0.75)(1.4,0.75)\rput{0}(0.7,0.55){$c_{1}$}
\psline[linecolor=blue]{<-]}(5.6,2.25)(7.0,2.25)\rput{0}(6.3,2.05){$c_{1}$}
\psline[linecolor=red]{<-]}(1.4,1.5)(2.8,1.5)\rput{0}(2.0999999999999996,1.3){$z_{100}$}
\psline[linecolor=red]{<-]}(0.0,2.25)(1.4,2.25)\rput{0}(0.7,2.05){$z_{100}$}
\pscircle[linecolor=red,fillcolor=black,fillstyle=solid](0.0,2.25){0.075}
\pscircle[linecolor=red,fillcolor=black,fillstyle=solid](1.4,1.5){0.075}
\pscircle[linecolor=red,fillcolor=white,fillstyle=solid](1.4,2.25){0.075}
\pscircle[linecolor=red,fillcolor=white,fillstyle=solid](2.8,1.5){0.075}
\psline[linecolor=red]{<-]}(2.8,0.75)(4.199999999999999,0.75)\rput{0}(3.4999999999999996,0.55){$z_{101}$}
\psline[linecolor=red]{<-]}(1.4,3.0)(5.6,3.0)\rput{0}(3.5,2.8){$z_{101}$}
\pscircle[linecolor=red,fillcolor=black,fillstyle=solid](1.4,3.0){0.075}
\pscircle[linecolor=red,fillcolor=black,fillstyle=solid](2.8,0.75){0.075}
\pscircle[linecolor=red,fillcolor=white,fillstyle=solid](5.6,3.0){0.075}
\pscircle[linecolor=red,fillcolor=white,fillstyle=solid](4.199999999999999,0.75){0.075}
\psline[linecolor=red]{<-]}(4.199999999999999,3.75)(5.6,3.75)\rput{0}(4.8999999999999995,3.55){$z_{102}$}
\pscircle[linecolor=red,fillcolor=black,fillstyle=solid](4.199999999999999,3.75){0.075}
\pscircle[linecolor=red,fillcolor=black,fillstyle=solid](5.6,4.5){0.075}
\pscircle[linecolor=red,fillcolor=white,fillstyle=solid](5.6,3.75){0.075}
\pscircle[linecolor=red,fillcolor=white,fillstyle=solid](7.0,4.5){0.075}
\psline[linecolor=red]{<-]}(5.6,4.5)(7.0,4.5)\rput{0}(6.3,4.3){$z_{102}$}
\end{pspicture}
\end{center}
{\bf GE Information}:  
Carrier: [0-1:z100-.] ;  
Carrier Dual: [1-2:z100-.] ;  
Critical Boundary: 1;  
Observe the following facts about this GE:
The base [2-3:z101-.]  and its dual are of the same polarity, yet one properly contains the other.  The base [1-4:z101-.]  and its dual are of the same polarity, yet one properly contains the other.  These observations show that the GE above is degenerate.  This GE is a leaf in the GE tree.  This branch of the tree has led us to a dead end.\\[0.1in]
This completes the consideration of root-6.1, as derived from the application of a print to root-6.\\[0.1in]
\section{Acknowledgements}
The authors acknowledge that this report was generated by software developed as part of a funded project supported by a research grant (H98230-06-1-0042) from the National Security Agency.  We also give special thanks to Alexei Miasnikov and Olga Kharlampovich for many helpful discussions along the way.  \end{document}
